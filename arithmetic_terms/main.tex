\pdfoutput=1
\documentclass[11pt,reqno]{article}
\usepackage{fullpage}
\usepackage{amsmath, amssymb, amsthm}
\usepackage[utf8]{inputenc}
\usepackage[english]{babel}
\usepackage[numbers]{natbib}
\usepackage[usenames]{color}
\usepackage{url}
\usepackage{color}
\usepackage[usenames]{color}
\usepackage[colorlinks=true,
linkcolor=webgreen,
filecolor=webbrown,
citecolor=webgreen]{hyperref}
\definecolor{webgreen}{rgb}{0,.5,0}
\definecolor{webbrown}{rgb}{.6,0,0}
\definecolor{red}{rgb}{1,0,0}
\usepackage{cleveref}
\usepackage{tabularx}
\usepackage{placeins}
\theoremstyle{plain}
\newtheorem{conjecture}{Conjecture}
\newtheorem{theorem}{Theorem}
\newtheorem{lemma}[theorem]{Lemma}
\newtheorem{corollary}[theorem]{Corollary}
\newtheorem{proposition}[theorem]{Proposition}
\newtheorem{remark}{Remark}
\crefname{conjecture}{Conjecture}{Conjectures}
\crefname{theorem}{Theorem}{Theorems}
\crefname{lemma}{Lemma}{Lemmas}
\crefname{proposition}{Proposition}{Propositions}
\crefname{remark}{Remark}{Remarks}
\crefname{corollary}{Corollary}{Corollaries}
\theoremstyle{definition}
\newtheorem{definition}{Definition}
\newtheorem{notation}{Notation}
\newtheorem{example}{Example}
\crefname{definition}{Definition}{Definitions}
\crefname{notation}{Notation}{Notations}
\crefname{example}{Example}{Examples}
\crefname{section}{\S}{Sections}
\newcommand{\floor}[1]{\left\lfloor #1 \right\rfloor}
\newcommand{\ceil}[1]{\left\lceil #1 \right\rceil}
\newcommand{\sg}{\mathrm{sg}}
\newcommand{\lcm}{\mathrm{lcm}}
\newcommand{\Z}{\mathbb{Z}}
\newcommand{\K}{\mathcal{K}}
\newcommand{\wt}[1]{\#(#1)}
\newcommand{\eval}[2]{\left . #1 \right|_{#2}}
\newcommand{\seqnum}[1]{\href{https://oeis.org/#1}{\rm \underline{#1}}}

\setlength{\parskip}{0.5em}
\setlength{\parindent}{0pt}

\title{Arithmetic Terms for Greatest Common Divisors and Semiprime Factors}
\author{Joseph M. Shunia}
\date{June 19, 2024}

\begin{document}

\maketitle

\begin{abstract}
We present new formulas for computing greatest common divisors (GCDs) and extracting the prime factors of semiprimes using only elementary arithmetic operations: addition, subtraction, multiplication, floored division, and exponentiation. Our GCD formula simplifies a result of Mazzanti, and is derived using Kronecker substitution techniques from our previous work. We utilize the GCD formula, along with recent developments on arithmetic terms for square roots and factorials, to derive explicit expressions for the prime factors of a semiprime $n=p_1p_2$.
\end{abstract}

\section{Introduction}

The greatest common divisor (GCD) of two integers $a$ and $b$, denoted $\gcd(a,b)$, is the largest positive integer that divides both $a$ and $b$. Euclid's algorithm for computing the GCD is one of the oldest known algorithms, dating back to ancient Greece \cite{knuth1997art}.

Semiprimes, which are numbers with exactly two prime factors, also play a key role in number theory and cryptography. The problem of factoring a semiprime $n=p_1 p_2$ into its constituent primes $p_1$ and $p_2$ is believed to be computationally intractable for large $n$ and forms the basis for widely used cryptosystems such as RSA \cite{rivest1978rsa}. Efficient algorithms for factoring semiprimes would have major implications for the security of these systems.

An ``arithmetic term'' is a mathematical expression which uses only the operations of addition $(a+b)$, subtraction $(a-b)$, multiplication $(ab)$, floored division $(\floor{a/b})$, and exponentiation $a^b$. Note that the modulo operation $(a \bmod b)$ is implied, since it can be expressed using subtraction and floored division:
\begin{align*}
    a \bmod b = a - b \floor{a/b} .
\end{align*}
Let $\textbf{A}$ denote the class of arithmetic terms. Formally, we have
\begin{align*}
\textbf{A} = [ \{ 1, a+b, a-b, ab, \floor{a/b}, a^b \} \} ].
\end{align*}

In this paper, we present new results on arithmetic term formulas for the GCD and semiprime factorization. Building on work by Mazzanti and Marchenkov \cite{mazzanti2002plainbases, marchenkov1980superposition}, we derive a simplified polynomial form for the GCD that can be expressed in terms of an arbitrary integer base. We also obtain arithmetic term formulas for the prime factors of a non-square semiprime $n=p_1 p_2$, using only the operations of addition, subtraction, multiplication, floored division, and exponentiation.

\subsection{Background}
Our new arithmetic term formulas for GCDs and semiprime factors, while entirely impractical, are of theoretical importance. We provide a brief overview and some historical context to demonstrate their significance.

\subsection{Kalmar Functions} \label{subsection:kalmar}
Firstly, we denote by $\textbf{P}$ the class of primitive recursive functions. The class of Kalmar functions, denoted by $\textbf{K}$, is an elementary class of functions, which is a subclass of $\textbf{P}$, defined as
\begin{align*}
    \textbf{K} = [\{ 1, a+b, a\Dot{-}b, ab, \floor{a/b}, a^b \} \}],
\end{align*}
where the notation $\Dot{-}$ represents so-called ``bounded'' or ``modified'' subtraction (See \cite{mazzanti2002plainbases} for a precise definition).

Kalmar functions were introduced by Laszlo Kalmar in the 1940s as a subclass \textbf{P}. Kalmar aimed to characterize the class of functions that can be computed using a certain restricted form of recursion, known as ``Kalmar elementary recursion'' or ``bounded recursion'' (hence the term ``bounded subtraction'' in the definition of $\textbf{K}$) \cite{herman1969elementary}. It is well-established that $\textbf{K}$ contains many important functions, such as the arithmetic operations, the exponential function, and the bounded $\mu$ operator (which is used to define the floored division operation). However, it does not contain all primitive recursive functions. For example, the well-known Busy Beavers Problem is primitive recursive, but is not Kalmar elementary.

It was long conjectured, and finally proved by Mazzanti, that the class $\textbf{A}$ generates the class $\textbf{K}$ \cite{mazzanti2002plainbases, marchenkov2007superposition}. As mentioned above, $\textbf{K}$ is known to be a proper subset of $\textbf{P}$. In summary, we have
\begin{align*}
    \textbf{A} \subseteq \textbf{K} \subset \textbf{P} .
\end{align*}

In 1970, Matiyasevich proved that all computable functions can be expressed as Diophantine equations \cite{matiyasevich1980diophantine, matiyasevich1993hilbert}. Matiyasevich's result implies that there exists a Diophantine equation for calculating the $n$-th prime number. However,  no arithmetic term for the $n$-th prime is known \cite{prunescu2024factorial}. Similarly, while Matiyasevich's theorem suggests the existence of an Diophantine equation formula for semiprime factorization, an arithmetic term that computes the factors remained to be discovered. Our work presents the first explicit arithmetic term formulas for this problem.

\subsection{Latest Discoveries}

Recently, Prunescu and Sauras-Altuzarra (2024) discovered an arithmetic term for computing the factorial function $n!$ \cite{prunescu2024factorial}. Coincidentally, at approximately the same time, we discovered an arithmetic term for the $n$-th roots of positive integers $\sqrt[n]{a}$ \cite{shunia2024polynomial}. By combining these results, along with a simplified version of Mazzanti's GCD formula (\cref{proof:mazzantigcd}), we obtain the first explicit arithmetic terms for semiprime factors. This answers a question from Shamir (1978), who first hypothesized the existence of such a formula when describing an algorithmic approach to integer factorization using arithmetic terms \cite{shamir1978factoring}.

\section{Greatest Common Divisor} \label{section:gcd}

\begin{lemma}[Mazzanti] \label{proof:mazzantigcd}
\begin{align*}
\forall a,b \in \Z^+, \quad
\gcd(a,b) = \floor{\frac{(2^{a^2 b(b+1)} - 2^{a^2 b}) (2^{a^2 b^2} - 1)}
         {(2^{a^2 b} - 1)(2^{ab^2}-1)2^{a^2 b^2}}} \bmod 2^{ab}.
\end{align*}
\end{lemma}
\begin{proof}
The lemma and proof belong to Mazzanti (2002) \cite{mazzanti2002plainbases}.
\end{proof}

Applying Kronecker substitution techniques from our previous works \cite{shunia2023simple, shunia2024polynomial}, we find that Mazzanti's formula can be simplified and expressed in a polynomial form.

\begin{theorem} \label{proof:gcdpolynomial}
\begin{align*}
\forall a,b \in \Z^+, \quad
\gcd(a,b) = \floor{\frac{x^{a+ab}}{(x^a - 1)(x^b-1)}}\bmod x .
\end{align*}
\end{theorem}
\begin{proof}
Consider Mazzanti's greatest common divisor formula (\cref{proof:mazzantigcd}), which is given by
\begin{align*}
\gcd(a,b) = \floor{\frac{(2^{a^2 b(b+1)} - 2^{a^2 b}) (2^{a^2 b^2} - 1)}
         {(2^{a^2 b} - 1)(2^{ab^2}-1)2^{a^2 b^2}}} \bmod 2^{ab}.
\end{align*}
Observe that all integer powers in the arithmetic term are divisible by $2^{ab}$. Factoring these, we obtain
\begin{align*}
\gcd(a,b) = \floor{\frac{((2^{ab})^{a (b+1)} - (2^{ab})^a) ((2^{ab})^{ab} - 1)}
 {((2^{ab})^{a} - 1)((2^{ab})^{b}-1) (2^{ab})^{ab}}} \bmod 2^{ab} .
\end{align*}
Substituting with $2^{ab} = x$ yields
\begin{align*}
\gcd(a,b) = \floor{\frac{(x^{a (b+1)} - x^a) (x^{ab} - 1)}
 {(x^{a} - 1)(x^{b}-1) x^{ab}}} \bmod x .
\end{align*}
The substitution is valid, since $2^{ab} > \gcd(a,b)$ and the substitution $2^{ab} = x$ essentially inverts the Kronecker substitution with the base $2^{ab}$ (See Theorem 1 in \cite{shunia2023simple}).

Simplifying the fraction, we see
\begin{align*}
\gcd(a,b) = \floor{
\frac
{ x^{a-ab} (x^{ab} - 1)^2 }
{ (x^a - 1)(x^b-1) }
}
 \bmod x .
\end{align*}
This fraction can be expanded as the sum
\begin{align*}
\gcd(a,b) = \floor{
    \frac
    { x^{a-ab} }
    { (x^a - 1)(x^b-1) }
    +
    \frac
    { x^{a+ab} }
    { (x^a - 1)(x^b-1) }
    +
    \frac
    { -2x^{a} }
    { (x^a - 1)(x^b-1) }
}
 \bmod x .
\end{align*}
Since we are reducing the quotient mod $x$, we need only consider the term in the fraction which yields the constant term in the polynomial, which is $\gcd(a,b)$. We find
\begin{align*}
\gcd(a,b) = \floor{\frac{x^{a+ab}}{(x^a-1)(x^b-1)}}\bmod x .
\end{align*}
\end{proof}

\begin{corollary} \label{proof:gcdintegerbase}
Let $a,b,n \in \Z^+$ such that $n > \gcd(a,b)$. Then
\begin{align*}
\gcd(a,b) = \floor{\frac{n^{a+ab}}{(n^a-1)(n^b-1)}}\bmod n .
\end{align*}
\end{corollary}
\begin{proof}
Consider the polynomial formula given by \cref{proof:gcdpolynomial}. Substituting with $x = n$ yields the given formula. By Theorem 2 in  \cite{shunia2024polynomial}, the substitution is valid since $n > \gcd(a,b)$.
\end{proof}

\section{Semiprime Factors} \label{section:semiprimes}
Using our results on the greatest common divisor function (\cref{section:gcd}), as well as results from our earlier works \cite{shunia2023simple,shunia2024polynomial} and those of Mazzanti \cite{mazzanti2002plainbases}, Prunescu and Sauras-Altuzarra \cite{prunescu2024factorial}, we discover arithmetic term formulas for the prime factors of a non-square semiprime $n=p_1 p_2$.

\begin{theorem} \label{proof:semiprimefactorp1}
Let $n \in \mathbb{Z}^+$ such that $n = p_1 p_2$ is a non-square semiprime and $p_1 < p_2$ are the prime factors of $n$.

Define
\begin{align*}
\omega = \floor{\frac{(n^{2n} + 1)^{2n+1} \bmod (n^{4n}-n)}{(n^{2n} + 1)^{2n} \bmod (n^{4n}-n)}} - 1 .
\end{align*}

Then, set
\begin{align*}
\gamma = \floor
{
    \frac{2^{\omega (\omega+1)(\omega+2)}}
    {
        \floor
        {
        (2^{2^{(\omega+1)(\omega+2)} - n} + 2^{-\omega})^{ 2^{(\omega+1)(\omega+2)} }
        }
        \bmod 2^{\omega 2^{(\omega+1)(\omega+2)} }
    }
} .
\end{align*}

Finally, we have
\begin{align*}
p_1 = \floor{\frac{n^{n+n\gamma}}{(n^n-1)(n^{\gamma}-1)}}\bmod n .
\end{align*}
\end{theorem}
\begin{proof}
From Shunia (2024) \cite{shunia2024polynomial}, for $n$ that is not a square, we get the arithmetic term
\begin{align*}
\floor{\sqrt{n}} &=
\floor{\frac{(n^{2n} + 1)^{2n+1} \bmod (n^{4n}-n)}{(n^{2n} + 1)^{2n} \bmod (n^{4n}-n)}} - 1 ,
\end{align*}
which matches our definition of $\omega$. Hence, $\omega = \floor{\sqrt{n}}$.

From Prunescu and Sauras-Altuzarra (2024) \cite{prunescu2024factorial}, we also have the factorial formula
\begin{align*}
n! &= \floor{2^{n(n+1)(n+2)} / \binom{2^{(n+1)(n+2)}}{n}} \\
&= \floor
{
    \frac
    {
        2^{n(n+1)(n+2)}
    }
    {
        \floor
        {
            \left(
                2^{2^{(n+1)(n+2)}-n} + 2^{-n}
            \right)^{2^{(n+1)(n+2)}}
        }
        \bmod
        2^{2^{(n+1)(n+2)}} .
    }
}
\end{align*}
Considering $\omega!$, this becomes
\begin{align*}
\omega! = \floor
{
    \frac{2^{\omega (\omega+1)(\omega+2)}}
    {
        \floor
        {
        (2^{2^{(\omega+1)(\omega+2)} - n} + 2^{-\omega})^{ 2^{(\omega+1)(\omega+2)} }
        }
        \bmod 2^{\omega 2^{(\omega+1)(\omega+2)} }
    }
} ,
\end{align*}
which matches the definition for $\gamma$. Hence, $\gamma = \omega! = \floor{\sqrt{n}}!$.

Applying \cref{proof:gcdintegerbase}, we have
\begin{align*}
    \gcd(n, \floor{\sqrt{n}}!) = \gcd(n,\gamma) = \floor{\frac{n^{n+n\gamma}}{(n^n-1)(n^{\gamma}-1)}}\bmod n .
\end{align*}
Since $n$ is a non-square semiprime and $p_1 < p_2$, we must have $p_1 < \floor{\sqrt{n}}$ and $p_2 > \floor{\sqrt{n}}$. Hence, $p_1 = \gcd(n,\floor{\sqrt{n}}!)$, which we showed is equivalent to the formula in the theorem.
\end{proof}
\begin{corollary} \label{proof:semiprimefactorp2}
\begin{align*}
p_2 = \frac{n}{\floor{\frac{n^{n+n\gamma}}{(n^n-1)(n^{\gamma}-1)}}\bmod n}.
\end{align*}
\end{corollary}
\begin{proof}
The proof follows immediately from \cref{proof:semiprimefactorp1}, since $\frac{n}{p_1} = p_2$ in this case.
\end{proof}

\begingroup
\raggedright
\bibliographystyle{unsrtnat}
\bibliography{main}
\endgroup

\end{document}