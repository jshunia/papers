\pdfoutput=1
\documentclass[12pt,reqno]{article}
\usepackage{fullpage}
\usepackage{amsmath, amssymb, amsthm}
\usepackage[utf8]{inputenc}
\usepackage[english]{babel}
\usepackage[numbers]{natbib}
\usepackage[usenames]{color}
\usepackage{url}
\usepackage{color}
\usepackage[usenames]{color}
\usepackage[usenames]{color}
\usepackage[colorlinks=true,
linkcolor=webgreen,
filecolor=webbrown,
citecolor=webgreen]{hyperref}
\definecolor{webgreen}{rgb}{0,.5,0}
\definecolor{webbrown}{rgb}{.6,0,0}
\definecolor{red}{rgb}{1,0,0}
\usepackage{cleveref}
\usepackage{tabularx}
\theoremstyle{plain}
\newtheorem{conjecture}{Conjecture}
\newtheorem{theorem}{Theorem}
\newtheorem{lemma}[theorem]{Lemma}
\newtheorem{corollary}[theorem]{Corollary}
\newtheorem{proposition}[theorem]{Proposition}
\newtheorem{remark}{Remark}
\crefname{conjecture}{Conjecture}{Conjectures}
\crefname{theorem}{Theorem}{Theorems}
\crefname{lemma}{Lemma}{Lemmas}
\crefname{proposition}{Proposition}{Propositions}
\crefname{remark}{Remark}{Remarks}
\theoremstyle{definition}
\newtheorem{definition}{Definition}
\newtheorem{notation}{Notation}
\newtheorem{example}{Example}
\crefname{definition}{Definition}{Definitions}
\crefname{notation}{Notation}{Notations}
\crefname{example}{Example}{Examples}
\crefname{section}{\S}{Sections}
\newcommand{\floor}[1]{\left\lfloor #1 \right\rfloor}
\newcommand{\Z}{\mathbb{Z}}
\newcommand{\K}{\mathcal{K}}
\newcommand{\wt}[1]{\#(#1)}
\newcommand{\eval}[2]{\left . #1 \right|_{#2}}
\newcommand{\seqnum}[1]{\href{https://oeis.org/#1}{\rm \underline{#1}}}

\setlength{\parskip}{0.5em}
\setlength{\parindent}{0pt}

\begin{document}
\title{Polynomial Quotient Rings and Kronecker Substitution for Deriving Combinatorial Identities}
\author{Joseph M. Shunia}
\date{March 2024}

\maketitle

\begin{abstract}
We establish a new connection between combinatorial number theory and polynomial ring theory by applying Kronecker substitution and related polynomial encoding techniques to polynomial expansions within quotient rings. We prove a new theorem which provides a general framework for generating combinatorial identities using polynomial quotient rings in conjunction with Kronecker substitution. We demonstrate the theorem's utility by deriving an explicit formula for famous integer sequences, such as the Fibonacci sequence, Pell sequence, and the central binomial coefficients. To give an example, we present a formula for the $n$-th Fibonacci number, valid for $n > 1$, given by: $F_n = (2^{n (n - 1)} \bmod{(4^n-2^n-1)}) \bmod{(2^n-1)}$. Additionally, we find a new formula for square roots using our approach. This work builds upon our previous results on binomial and multinomial coefficients, extending the application of Kronecker substitution beyond its traditional use in improving the efficiency of integer and polynomial multiplication algorithms.
\end{abstract}

\section{Introduction}
Kronecker substitution, named after the mathematician Leopold Kronecker, is a technique that allows for the efficient multiplication of integers and polynomials by encoding them as integers in a larger base \cite{kronecker1882}. While this technique has been widely used in the design of fast multiplication algorithms \cite{harvey2009kronecker, harvey2014faster, albrecht2018implementing, bos2020postquantum, greuet2022modular}, its potential applications in combinatorial number theory have remained largely unexplored.

In an unpublished preprint \cite{shunia2023simple}, we took the first steps in this direction by applying Kronecker substitution to binomial expansions, yielding a new formula for binomial coefficients:
\begin{align*}
    \binom{n}{k} = \left\lfloor \frac{(2^n+1)^n}{2^{n k}} \right\rfloor \bmod{2^n}
\end{align*}

In this work, we develop a general framework for applying Kronecker substitution to polynomial expansions within quotient rings, which is useful for generating combinatorial identities. Our main theorem establishes a connection between the coefficients of a polynomial remainder and the integers obtained by evaluating the polynomials at specific values. By carefully selecting these values, we can generate new identities for combinatorial sequences.

\subsection{New Formulas}
To demonstrate the power of our approach, we apply our main theorem to derive new explicit formulas for several important combinatorial sequences. We also find a result on square roots.

First, we obtain a formula for the Fibonacci sequence, one of the most well-known and widely studied combinatorial sequences. The Fibonacci sequence is \seqnum{A000045} in the OEIS \cite{A000045}. Valid for $n > 1$, our formula expresses the $n$-th Fibonacci number $F_n$ in terms of a double modular expression involving powers of $2$:
\begin{align*}
    F_n = (2^{n(n-1)} \bmod{(4^n-2^n-1)}) \bmod{(2^n-1)}
\end{align*}

Second, we derive a new formula for the Pell sequence, which is another fundamental integer sequence. The Pell sequence is \seqnum{A000129} in the OEIS \cite{A000129}. Valid for $n > 1$, our formula expresses the $n$-th Pell number $P_n$ in terms of a double modular expression involving powers of $2$:
\begin{align*}
    P_n = ((2^{n}+1)^{n-1} \bmod{(4^n-2)}) \bmod{(2^n-1)}
\end{align*}

In proving the above formula for the Pell sequence, we uncover a general formula for square roots. Let $n \in \mathbb{Z}^+$ such that $n > 1$. Then
\begin{align*}
\sqrt{n} &= \lim_{k\rightarrow\infty} \quad
    \frac{n^k ((n^k + 1)^k \bmod{(n^{2k}-n, n^k)})}
        {(n^k + 1)^k \bmod{(n^{2k}-n)}}
\end{align*}

Finally, we derive a new formula for the central binomial coefficients, which are the binomial coefficients of the form $\binom{2n}{n}$. These coefficients form sequence \seqnum{A000984} in the OEIS \cite{A000984} and have numerous applications in combinatorics and number theory. Our formula, which is valid for $n > 0$, expresses the $n$-th central binomial coefficient in terms of a double modular expression involving powers of $4$:
\begin{align*}
    \binom{2n}{n} &= ((4^n + 1)^{2n} \bmod{(4^{n(n+1)} + 1)}) \bmod{(4^n-1)}
\end{align*}

\subsection{Structure of the Paper}
The rest of this paper is organized as follows. In \cref{section:notations}, we introduce the notations used throughout the paper. \cref{section:kronecker} provides a brief primer on Kronecker substitution. Our main results, including the quotient ring encoding theorem and its proof, are presented in \cref{section:results}. In \cref{section:applications}, we apply our quotient ring encoding theorem to derive new formulas. We follow up with \cref{section:modn}, in which we describe a method for computing our formulas in $\mathbb{Z}/n\mathbb{Z}$. We conclude with \cref{section:futureresearch} by detailing some of our plans for future research in this area.

\section{Notations} \label{section:notations}
This section provides a brief overview of the notations used throughout this paper.

\begin{notation}[Sequential moduli]
Let $n \in \mathbb{Z}$ and let $(m_0, m_1, \ldots, m_k)$ be a sequence of moduli. We define the application of mod operations on $n$ by this sequence as follows:
\begin{align*}
    n \bmod{(m_0, m_1, \ldots, m_k)} \Longleftrightarrow  (((n \bmod{m_0}) \bmod{m_1}) \cdots ) \bmod{m_k},
\end{align*}
where the mod operations are performed sequentially from left to right, following the order of the moduli as listed.
\end{notation}

\begin{notation}[Polynomial normalized form]

Given a polynomial of the form
\begin{align*}
f(x) = a_d x^d + a_{d-1} x^{d-1} + \cdots + a_0
\end{align*}
We use the notation $\tilde{f}(x)$ to represent its normalized form, where the leading coefficient is scaled to $1$. Formally, we can write this as
\begin{align*}
    \tilde{f}(x) = \frac{f(x)}{a_d} = x^d + \frac{a_{d-1}}{a_d} x^{d-1} + \cdots + \frac{a_0}{a_d}
\end{align*}
\end{notation}

\section{A Brief Primer on Kronecker Substitution} \label{section:kronecker}
Kronecker substitution, named after the mathematician Leopold Kronecker who first described it in 1882 \cite{kronecker1882}, is a technique for converting a polynomial to an integer representation. Given a polynomial $f(x) \in \mathbb{Z}[x]$ and a suitable integer $b \in \mathbb{Z}$, Kronecker substitution evaluates $f(x)$ at $x = b$. By choosing an appropriate base $b$, the resulting integer $f(b)$ encodes the coefficients of $f(x)$ in its digits.

More formally, let $f(x) \in \mathbb{Z}[x]$ be a polynomial of degree $d$, represented as
\begin{align*}
f(x) = a_d x^d + a_{d-1} x^{d-1} + \cdots + a_1 x + a_0,
\end{align*}
where $a_i \in \mathbb{Z}$ for $0 \leq i \leq d$. Performing Kronecker substitution with $x = b$ yields the integer
\begin{align*}
f(b) = a_d b^d + a_{d-1} b^{d-1} + \cdots + a_1 b + a_0.
\end{align*}

When $b$ is sufficiently large, the base-$b$ representation of $f(b)$ directly corresponds to the coefficients of $f(x)$. In other words, the digits of $f(b)$ in base $b$ are precisely the coefficients $a_d, a_{d-1}, \ldots, a_1, a_0$, in order from most significant to least significant.

To ensure a one-to-one correspondence between the coefficients and the digits, the base $b$ must be chosen such that
\begin{align*}
    b > \max_{0 \leq i \leq d} |a_i|
\end{align*}
This guarantees that there is no ``carry over'' between digits when performing arithmetic operations on the integer representation.

\begin{remark}
Traditionally, the base $b$ is selected to be the smallest power of $2$ which is greater than any of the coefficients of the encoded polynomial. Though, this is typically not the optimally smallest base. In \cref{theorem:encoding}, we show that $b=f(1)$ is optimal for any polynomial in $\mathbb{Z}[x]$ with non-negative coefficients.
\end{remark}

The process of Kronecker substitution can be reversed to recover the original polynomial $f(x)$ from its integer representation $f(b)$. Given $f(b)$ and the base $b$, one can extract the coefficients by successively dividing $f(b)$ by powers of $b$ and taking the remainders. This allows for the reconstruction of $f(x)$ from $f(b)$ \cite{grimaldi2004discrete}.

Kronecker substitution has found numerous applications in computer algebra and symbolic computation, particularly in the design of efficient algorithms for polynomial multiplication \cite{harvey2009kronecker, harvey2014faster}. By reducing polynomial operations to integer arithmetic, Kronecker substitution enables the use of fast integer multiplication algorithms, resulting in improved performance for polynomial computations.

\subsection{Kronecker Substitution in Action}
To demonstrate Kronecker substitution, we proceed with a simple example in base-$10$.

\begin{example}
Fix $b=10$. Consider the polynomial 
\begin{align*}
    f(x) = 4x^3 + 3x^2 + 2x + 1 \quad \in \mathbb{Z}[x]
\end{align*}

\textbf{Step 1. Encoding:} \\
We encode the coefficients of $f(x)$ using base $b=10$, chosen because it exceeds the absolute values of all coefficients in $f(x)$, thereby ensuring no overlap or ``carry over'' in the encoding process.

Substituting $x = 10$ into $f(x)$, we calculate
\begin{align*}
    f(10) &= 4(10)^3 + 3(10)^2 + 2(10) + 1 \\
    &= 4000 + 300 + 20 + 1 \\
    &= 4321
\end{align*}

\textbf{Step 2. Decoding:} \\
To decode $f(x)$ from its integer representation, $f(10) = 4321$, one can extract each digit according to its positional value in base-$10$:

\begin{itemize}
    \item The thousands digit, $4$, is the coefficient of $x^3$.
    \item The hundreds digit, $3$, is the coefficient of $x^2$.
    \item The tens digit, $2$, is the coefficient of $x$.
    \item The units digit, $1$, is the constant term.
\end{itemize}

Thus, we can reconstruct $f(x) = 4x^3 + 3x^2 + 2x + 1$ from the integer $4321$ by interpreting each digit according to its position in the base-$10$ representation.
\end{example}

\section{Polynomial Encoding Theorem}
For completeness, we restate a theorem we first shared in our previous work on a formula for binomial coefficients \cite{shunia2023simple}.

\begin{theorem} \label{theorem:encoding}
Let $b, d \in \mathbb{Z}$ such that $b > 0$, $d \geq 0$. Consider a polynomial $f(x) \in \mathbb{Z}[x]$ of degree $d$, which has the form
\begin{align*}
    f(x) = a_d x^d + a_{d-1} x^{d-1} + \cdots + a_1 x + a_0
\end{align*}
Suppose $f(b) \not= 0$ and that all coefficients of $f(x)$ are non-negative. Then, $f(x)$ can be completely determined by the evaluations $f(b)$ and $f(f(b))$. Furthermore, the coefficient $a_k$, where $0 \leq k \leq d$, can be recovered explicitly using the formula:
\begin{align*}
a_k = \floor{\frac{f(f(b)) \bmod{f(b)^{k + 1}}}{f(b)^{k}}}
\end{align*}
\end{theorem}
\begin{proof}
By assumption, for all coefficients $a_k$ of $f(x)$, where $0 \leq k \leq d$, we can recover $a_k$ using the given formula.

To prove the validity of the formula, we proceed by examining its arithmetic operations step-by-step. Let's consider the expansion of $f(f(b))$, which is
\begin{align*}
f(f(b)) &= a_d f(b)^d + a_{d-1} f(b)^{d-1} + \cdots + a_1 f(b) + a_0
\end{align*}

The first step in the formula is to take $f(f(b)) \bmod{f(b)^{k + 1}}$. In doing so, we effectively isolate the terms up to $x^k$. The result is
\begin{align*}
f(f(b)) \bmod{f(b)^{k + 1}} &= (a_d f(b)^d + a_{d-1} f(b)^{d-1} + \cdots + a_1 f(b) + a_0) \bmod{f(b)^{k + 1}} \\
&= a_k f(b)^k + \cdots + a_1 f(b) + a_0 
\end{align*}

The next step is to divide by $f(b)^k$, which gives
\begin{align*}
\frac{f(f(b))  \bmod{f(b)^{k + 1}}}{f(b)^{k}} &= f(b)^{-k} (a_k f(b)^k + \cdots + a_1 f(b) + a_0) \\
&= a_k f(b)^{k-k} + a_k f(b)^{k-(k-1)} + \cdots + a_1 f(b)^{1-k} + a_0 b^{-k} \\
&= a_k + \cdots + a_1 f(b)^{1-k} + a_0 b^{-k}
\end{align*}

Finally, since $b > |a_j|$ for all $j$ in $0 \leq j \leq (k-1)$, the floor operation isolates the coefficient we want
\begin{align*}
\floor{\frac{f(f(b)) \bmod{f(b)^{k + 1}}}{f(b)^{k}}}
&= \floor{a_k + \cdots + a_1 f(b)^{1-k} + a_0 b^{-k}} \\
&= a_k
\end{align*}

The floor operation ensures that the result is in $\mathbb{Z}$, and since the coefficients are non-negative, the floor operation does not affect the result.

Thus, we can recover all the coefficients $a_0, a_1, \ldots, a_d$ using only the values $f(b)$ and $f(f(b))$. Furthermore, since we can recover $a_k$ given $k$, we can determine the degree of the term corresponding to the coefficient recovered.

In conclusion, under the given conditions, $f(x)$ can be completely determined by the evaluations $f(b)$ and $f(f(b))$ using the provided formula.
\end{proof}

\section{Main Result} \label{section:results}
For our main result, we apply $\cref{theorem:encoding}$ to polynomial quotient rings to devise a theorem which serves a framework for translating polynomial quotient rings to combinatorial identities.

\begin{theorem} \label{theorem:kroneckerqrings}
Let $k, d \in \mathbb{Z}^+$ such that $k \geq d$. Consider a polynomial
\begin{align*}
g(x) = a_d x^d - a_{d-1} x^{d-1} - \cdots - a_0 \in \mathbb{Z}[x]
\end{align*}
and the remainder
\begin{align*}
r(x) = f(x)^k  \bmod{\tilde{g}(x)} ,
\end{align*}
where $f(x)$ is any non-constant polynomial in $\mathbb{Z}[x]$. Let $b, \gamma \in \mathbb{Z}^+$ and suppose $\gamma^k \geq |r(b)|$. Then,
\begin{align*}
r(b) = f(\gamma^k)^k \bmod{(\tilde{g}(\gamma^k), \gamma^k - b)}
\quad \textup{ or }
\quad r(b) \equiv 0 \pmod{\gamma^k - b}
\end{align*}

\end{theorem}
\begin{proof}
First, consider the evaluation 
\begin{align*}
    \eval{r(x)}{x=b} = r(b)
\end{align*}
In modular arithmetic, evaluating a polynomial $h(x) \in \mathbb{Z}[x]$ at $x=b$ is the same as taking $h(x)$ modulo $(x - b)$. In our case, since we are working modulo $\tilde{g}(x)$, we have the relation
\begin{align*}
    r(b) = f(x)^k \bmod{(\tilde{g}(x), x - b)}
\end{align*}

Applying Kronecker substitution to all polynomials in the above equation, using the substitution $x = \gamma^k$, yields
\begin{align*}
    r(b) = f(\gamma^k)^k \bmod{(\tilde{g}(\gamma^k), \gamma^k - b)}
\end{align*}

Which is the formula we aimed to prove. Furthermore, recall that we are given $\gamma$ such that
\begin{align*}
    \gamma^k \geq |r(1)|
\end{align*}
This implies that, when applying Kronecker substitution, the base $\gamma^k$ is sufficient to losslessly encode all of the coefficients of $r(x)$ (\cref{theorem:encoding}). Moreover, since $k \geq d$, the same is true of $f(x)$ and $g(x)$. Thus, the only way to have
\begin{align}
    r(b) \not= f(\gamma^k)^k \bmod{(\tilde{g}(\gamma^k), \gamma^k - b)} ,
\end{align}
is if
\begin{align*}
    & (\gamma^k - b) \mid (f(\gamma^k)^k \bmod{(\tilde{g}(\gamma^k)}) \\
    & \Longleftrightarrow (\gamma^k - b) \mid r(b) \\
    & \Longleftrightarrow r(b) \equiv 0 \pmod{\gamma^k - b}
\end{align*}
This completes the proof.
\end{proof}

\section{Applications} \label{section:applications}

\subsection{Fibonacci Formula}
To demonstrate the practical applications of \cref{theorem:kroneckerqrings}, we apply it to derive a new formula for the $n$-th Fibonacci number, which is sequence \seqnum{A000045} in the OEIS \cite{A000045}.  Starting from $n=1$, the sequence $F_n$ begins as
\begin{align*}
    F_n = 0, 1, 1, 2, 3, 5, 8, 13, 21, 34, 55, 89, 144, 233, 377, 610, 987, 1597, 2584, 4181, 6765, \ldots
\end{align*}

\begin{theorem} \label{theorem:fibonacci}
Let $F_n$ denote the $n$-th term of the Fibonacci sequence, such that $F_0 = 0, F_1 = 1$, and for $n > 1$:
\begin{align*}
    F_n = F_{n-1} + F_{n-2}
\end{align*}

Then, for $n > 1$
\begin{align*}
F_n &= 2^{n (n - 1)} \bmod{(4^n-2^n-1, 2^n-1)}
\end{align*}
\end{theorem}
\begin{proof}
Fix a ring $R = \mathbb{Z}[x]/(x^2 - x - 1)$. In the ring $R$, the elements obey the relation $x^2 = x + 1$. Solving for $x$ using the quadratic equation gives the solutions
\begin{align*}
    x = \frac{1 + \sqrt{5}}{2}, \quad x = \frac{1 - \sqrt{5}}{2}
\end{align*}

Since $F_n$ is always non-negative for $n \geq 0$, we choose $x = \frac{1 + \sqrt{5}}{2} = \varphi$, where $\varphi$ denotes the so-called ``golden ratio''. From the OEIS \cite{A000045}, we have the following formula:
\begin{align*}
    \varphi^{n-1} = F_{n-1} \varphi + F_{n-2}
\end{align*}

Substituting $\varphi = x \in R$, we can see
\begin{align*}
    (F_{n-1} x + F_{n-2}) \bmod{(x-1)} = x^{n-1} \bmod{(x^2 - x - 1, x-1)}
\end{align*}

Applying \cref{theorem:kroneckerqrings} by substituting with $x = 2^n$ and simplifying, yields
\begin{align*}
    (F_{n-1} x + F_{n-2}) \bmod{(x-1)} &= (2^n)^{n-1} \bmod{((2^n)^2 - 2^n - 1, 2^n-1)} \\
    F_n &= 2^{n (n - 1)} \bmod{(4^n - 2^n - 1, 2^n-1)}
\end{align*}
By \cref{theorem:kroneckerqrings}, this substitution is valid since $2^n \geq F_n$.

Considering $n = 1$, since $2^1-1 = F_1 = 1$, we have $F_1 \equiv 0 \pmod{2^n - 1}$. Thus, the formula is valid for $n > 1$.
\end{proof}

\subsection{Pell Formula}
We apply \cref{theorem:kroneckerqrings} to derive a new formula for the $n$-th Pell number, which is sequence \seqnum{A000129} in the OEIS \cite{A000129}. Starting from $n=1$, the sequence $P$ begins as
\begin{align*}
    P_n = 0, 1, 2, 5, 12, 29, 70, 169, 408, 985, 2378, 5741, 13860, 33461, 80782, 195025, \ldots
\end{align*}

\begin{theorem} \label{theorem:pell}
Let $P_n$ denote the $n$-th term of the Pell sequence, such that $P_0 = 0, P_1 = 1$, and for $n > 1$:
\begin{align*}
    P_n = 2 P_{n-1} + P_{n-2}
\end{align*}

Then, for $n > 1$
\begin{align*}
P_n &= (2^n + 1)^{n-1} \bmod{(4^n-2, 2^n-1)}
\end{align*}
\end{theorem}
\begin{proof}
Fix a ring $R = \mathbb{Z}[x]/(x^2 - 2)$. Consider $f(x) = x+1 \in R$. Expanding $f(x)^n$ using the binomial theorem gives
\begin{align*}
    f(x)^n = (x+1)^n = \sum_{k=0}^{n} \binom{n}{k} x^n
\end{align*}

In the ring $R$, the elements obey the relation $x^2 = 2$. This means that in the ring $R$, all elements are implicitly reduced modulo $(x^2 - 2)$. Factoring out $x^2$ from $x^k$ in our previous expansion (to apply the modular reduction), and then simplifying, yields
\begin{align*}
    f(x)^n \bmod{(x^2 - 2)} = \sum_{k=0}^{n} \binom{n}{k} 2^{\floor{\frac{1}{2} k}} x
    = \sum_{k=0}^{n} \binom{n}{k} 2^{\floor{\frac{k}{2}}} x
\end{align*}

Recall that evaluating at $x=1$ is the same as reducing modulo $(x-1)$. Thus, we have
\begin{align*}
    f(x)^n \bmod{(x^2-2, x-1)} = \sum_{k=0}^{n} \binom{n}{k} 2^{\floor{\frac{k}{2}}}
\end{align*}

From the OEIS \cite{A000129}, we have the formula
\begin{align*}
    P_{n+1} = \sum_{k=0}^{n} \binom{n}{k} 2^{\floor{\frac{k}{2}}}
\end{align*}

Hence
\begin{align*}
    P_{n+1} &= f(x)^n \bmod{(x^2-2, x-1)} \\
    &= (x+1)^n \bmod{(x^2-2, x-1)}
\end{align*}

Replacing $n$ with $n-1$ yields
\begin{align*}
    P_n &= (x+1)^{n-1} \bmod{(x^2-2, x-1)}
\end{align*}

To arrive at our stated formula, we apply \cref{theorem:kroneckerqrings} by substituting with $x = 2^n$, followed by simplifying
\begin{align*}
     P_n &= (2^n+1)^{n-1} \bmod{((2^n)^2-2, 2^n-1)} \\
     &= (2^n+1)^{n-1} \bmod{(4^n-2, 2^n-1)}
\end{align*}
By \cref{theorem:kroneckerqrings}, this substitution is valid since $2^n \geq P_n$.

Considering $n = 1$, we have $P_1 = 1$. However, in applying our derived formula, we arrive at a contradiction
\begin{align*}
     P_1 &= (2^1+1)^{1-1} \bmod{(4^1-2, 2^1-1)} & \\ 
     P_1 &= 3^{0} \bmod{(2, 1)} & \\
     P_1 &= 1 \bmod{(2, 1)} & \\
     P_1 &= 0 & \\
     \Rightarrow 1 &= 0 & \text{(contradiction)}
\end{align*}
Hence, the formula is invalid for $n=1$. On the other hand, considering $n > 1$, we see that
\begin{align*}
     & (x+1)^n \not\equiv 0 \pmod{x^2-2, x-1}, \quad \forall n > 1 ,
\end{align*}
leading to
\begin{align*}
    & (2^n+1)^n \not\equiv 0 \pmod{4^n-2, 2^n-1}, \quad \forall n > 1 .
\end{align*}
Thus, the given formula is valid for $n > 1$.
\end{proof}

\subsection{Central Binomial Coefficients Formula}
We apply \cref{theorem:kroneckerqrings} to derive a new formula for the $n$-th central binomial coefficient $\binom{2n}{n}$, which is sequence \seqnum{A000984} in the OEIS \cite{A000984}. Starting from $n=1$, the sequence of central binomial coefficients begins as
\begin{align*}
    \binom{2n}{n} = 1, 2, 6, 20, 70, 252, 924, 3432, 12870, 48620, 184756, 705432, 2704156, 10400600, \ldots
\end{align*}

\begin{theorem} \label{theorem:cbc}
Let $n \in \mathbb{Z}^+$ such that $n > 0$. Then
\begin{align*}
\binom{2n}{n} &= (4^n + 1)^{2n} \bmod{(4^{n(n+1)} + 1, 4^n - 1)}
\end{align*}
\end{theorem}
\begin{proof}
Fix a ring $R = \mathbb{Z}[x]/(x^{n+1} + 1)$. In the ring $R$, the elements obey the relation $x^{n+1} = -1$.

Let $f(x) = (x + 1)^{2n} \in R$. Expanding $f(x)$ and taking the result modulo $(x-1)$ gives
\begin{align*}
& (x + 1)^{2n} \bmod{(x^{n+1} + 1, x - 1)} \\
&= \sum_{k=0}^{2n} \binom{2n}{k} (-1)^{\floor{\frac{k}{n+1}}} \\
&= \left( \sum_{k=0}^{n} \binom{2n}{k} (-1)^{\floor{\frac{k}{n+1}}} \right) + \left( \sum_{k=n+1}^{2n} \binom{2n}{k} (-1)^{\floor{\frac{k}{n+1}}} \right) \\
&= \left( \sum_{k=0}^{n} \binom{2n}{k} (-1)^0 \right) + \left( \sum_{k=n+1}^{2n} \binom{2n}{k} (-1)^1 \right) \\
&= \left( \sum_{k=0}^{n} \binom{2n}{k} \right) - \left( \sum_{k=n+1}^{2n} \binom{2n}{k} \right) \\
&= \binom{2n}{n}
\end{align*}

Thus, we have
\begin{align*}
    \binom{2n}{n} &= (x + 1)^{2n} \bmod{(x^{n+1} + 1, x - 1)}
\end{align*}

Applying \cref{theorem:kroneckerqrings} by substituting with $x = 4^n$ and simplifying, yields
\begin{align*}
    \binom{2n}{n} &= (4^n + 1)^{2n} \bmod{(4^{n(n+1)} + 1, 4^n-1)}
\end{align*}
By \cref{theorem:kroneckerqrings}, this substitution is valid since $4^n \geq \binom{2n}{n}$.

Considering $n = 0$, since $4^0-1 = 0$, the final modulus in the sequence $(4^{n(n+1)} + 1, 4^n-1)$ is undefined. Thus, the formula is valid for $n > 0$.
\end{proof}

\subsection{Square Roots}
The previous result on Pell numbers (\cref{theorem:pell}) leads us to an interesting result on square roots.

\begin{theorem} \label{theorem:squareroots}
Let $n \in \mathbb{Z}^+$ such that $n > 1$. Then
\begin{align*}
\sqrt{n} &= \lim_{k\rightarrow\infty} \quad
    \frac{n^k ((n^k + 1)^k \bmod{(n^{2k}-n, n^k)})}
        {(n^k + 1)^k \bmod{(n^{2k}-n)}}
\end{align*}
\end{theorem}
\begin{proof}
Fix a ring $R = \mathbb{Z}[x]/(x^2 - n)$. In the ring $R$, the elements obey the relation $x^2 = n$. This means that in the ring $R$, all elements are implicitly reduced modulo $(x^2 - n)$.

Consider $f(x) = x+1 \in R$. Let $a,b \in \mathbb{Z}$. Expanding $f(x)^n$ in $R$ yields a polynomial of the form
\begin{align*}
f(x)^n \equiv a + b x \pmod{x^2 - n}
\end{align*}

Let $g(x) = f(x) \bmod{(x^2-n)}$. To find $\sqrt{n}$, it suffices to solve for $x$. But first, we must solve for the two unknowns $a,b$.

To solve for $a$, we simply need to find the constant term in $g(x)$. To do so, we can simply consider $g(x)$ modulo $x$, which gives
\begin{align*}
a = g(x) \bmod{x}
\end{align*}

Solving for $b$ is also simple. We need to find the coefficient $[x] g(x)$. To do so, we divide $g(x)$ by $x$ to get
\begin{align*}
b = x^{-1} g(x)
\end{align*}

Now, solving for $x$ in $a + b x$ yields
\begin{align*}
x = -a b^{-1}
\end{align*}

Here, $x$ is likely to be irrational. Hence, the solution to $x$ will not be exact, and will grow more precise as $n$ goes to infinity. Taking the limit, and substituting for all variables in $x = -a b^{-1}$, gives
\begin{align*}
x &= \lim_{k\rightarrow\infty} \quad
    \frac{-x ((x + 1)^k \bmod{(x^{2}-n, x)})}
        {(x + 1)^k \bmod{(x^{2}-n)}}
\end{align*}

Next, we apply Kronecker substitution by substituting for $x = n^k$ on the right-hand side to get
\begin{align*}
x &= \lim_{k\rightarrow\infty} \quad
    \frac{-n^k ((n^k + 1)^k \bmod{(n^{2k}-n, n^k)})}
        {(n^k + 1)^k \bmod{(n^{2k}-n)}}
\end{align*}
By \cref{theorem:kroneckerqrings}, this substitution is valid since $n^k \geq f(x)^k \bmod{(x^2-n,x-1)}$.

Typically, $\sqrt{n}$ is defined as positive. Making this adjustment, and substituting with $x = \sqrt{n}$ on the left-hand side, gives
\begin{align*}
\sqrt{n} &= \lim_{k\rightarrow\infty} \quad
    \frac{n^k ((n^k + 1)^k \bmod{(n^{2k}-n, n^k)})}
        {(n^k + 1)^k \bmod{(n^{2k}-n)}}
\end{align*}
Which is the formula we wanted prove.
\end{proof}

\section[Kronecker Substitution Mod n]{Kronecker Substitution Mod $n$} \label{section:modn}
As it is relevant to our applied results (\cref{section:applications}), we will demonstrate how Kronecker substitution can be efficiently performed in $\mathbb{Z}/n\mathbb{Z}$. Typically, mod $n$ reductions in Kronecker substitution are performed in a ``delayed reduction'' process \cite{greuet2022modular, dumas2011simulataneous}, which can be described as follows:

Given a polynomial encoded as an integer, the first step is decode the polynomial from its integer representation. Next, the decoded polynomial is reduced mod $n$. Finally, the polynomial remainder is encoded back to an integer in the chosen base.

While this algorithmic approach is considered sufficient (perhaps even ideal) for the purposes of fast multiplication algorithms, it is not ideal for us. We desire to derive explicit combinatorial identities and formulas using modular exponentiation, which necessitates in-place reduction of the coefficients mod $n$. For this purpose, we offer the theorem below.

\begin{theorem} \label{theorem:modn}
Let $n, b, d \in \mathbb{Z}$ such that $b > n > 0$ and $d \geq 0$. Consider a polynomial $f(x) \in \mathbb{Z}[x]$ of degree $d$, which has the form
\begin{align*}
    f(x) = a_d x^d + a_{d-1} x^{d-1} + \cdots + a_1 x + a_0
\end{align*}

Fix a polynomial 
\begin{align*}
    m(x) = \frac{n x^d - n}{x-1} = n x^d + n x^{d-1} + \cdots + n x + n
\end{align*}

Suppose $b > n$. Then, the polynomial $f(x) \pmod{n}$ can be recovered completely by calculating the integer remainder
\begin{align*}
    r \equiv f(b) \pmod{m(b)}
\end{align*}
and then reading the coefficients of $r$ out in base $b$.
\end{theorem}
\begin{proof}
First, observe that evaluating $m(x)$ at $x=b$ yields
\begin{align*}
m(b) = n b^d + n b^{d-1} + \cdots + n b + n
\end{align*}
resulting in an integer where each digit in base $b$ is $n$, given the condition that $b > n$.

Now, consider the operation of reducing $f(x) \bmod{n}$. We begin with
\begin{align*}
    f(x) \bmod{n} = (a_d x^d + a_{d-1} x^{d-1} + \cdots + a_1 x + a_0) \bmod{n}
\end{align*}

Distributing the mod operation, we see
\begin{align*}
    f(x) \bmod{n} = (a_d x^d \bmod{n}) + (a_{d-1} x^{d-1} \bmod{n}) + \cdots + (a_1 x \bmod{n}) + (a_0 \bmod{n})
\end{align*}

The above polynomial process is analogous to the process of reducing our encoded integer value $f(b)  \bmod{m(b)}$. Writing those integers in expanded form
\begin{align*}
    f(b) &= a_d b^d + a_{d-1} b^{d-1} + \cdots + a_1 b + a_0 \\
    m(b) &= n b^d + n b^{d-1} + \cdots + n b + n
\end{align*}

Likewise, expanding the integer remainder $r$, we have
\begin{align*}
    r &\equiv f(b) \pmod{m(b)} \\
    &\equiv (a_d b^d + a_{d-1} b^{d-1} + \cdots + a_1 b + a_0) \bmod{m(b)}
\end{align*}

Due to the nature of integer arithmetic and our encoding scheme, the distributed mod operation under $f(b) \bmod{m(b)}$ reduces to
\begin{align*}
    r &\equiv ((a_d b^d \bmod{(n b^d)}) + \cdots + (a_1 b \bmod{(n b)}) + (a_0 \bmod{n})) \bmod{m(b)}
\end{align*}
To see why, consider that all digits in $f(b)$ are reduced modulo $n$ at their corresponding digit place. Thus, any remainders are also reduced modulo $n$ at their corresponding digit places. The recursive chaining of these reductions simplifies the operation to reducing the individual base $b$ digits modulo $n$.

Hence, the integer value $r$ encodes the coefficients of $f(x) \pmod{n}$. By applying Kronecker substitution, we may recover those coefficients along with their degree. This completes the proof.
\end{proof}

The method described by \cref{theorem:modn} is not entirely novel. Dumas et al. \cite{dumas2011simulataneous} give a method for performing a ``simultaneous reduction'' using floored divisions which avoids the intermediate conversion steps in the ``delayed reduction''. However, their approach is still algorithmic in nature and therefore not applicable for our purposes.

\section{Future Research} \label{section:futureresearch}
Harvey (2009) \cite{harvey2009kronecker} proposed a method for improving the efficiency of integer multiplication by performing Kronecker substitution via multiple evaluations, termed ``multi-point Kronecker substitution''. Harvey described how to compute his method up to $r=4$ points. Bos et al. (2020) \cite{bos2020postquantum} generalized Harvey's result using an approach which they have called ``Kronecker+''. A brief description of the process is as follows:

First, a base $b$ is selected along with a fixed constant $r$, which corresponds to the number of evaluation points. Next, the polynomial $f(x)$ is evaluated at $r$ points, which are the products of $b$ times the $r$th roots of unity. The results of all evaluations are then combined to recover the polynomial coefficients. This process is essentially a variation of the Discrete Fourier Transform (DFT) applied to Kronecker substitution.

The aforementioned techniques both present promising avenues for future research.

\begingroup
\raggedright
\bibliographystyle{unsrtnat}
\bibliography{main}
\endgroup

\end{document}