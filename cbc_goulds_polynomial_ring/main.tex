\pdfoutput=1
\documentclass{article}
\usepackage{fullpage}
\usepackage{amsmath, amssymb, amsthm}
\usepackage[utf8]{inputenc}
\usepackage[english]{babel}
\usepackage[numbers]{natbib}
\usepackage{csquotes}
\usepackage{url}
\usepackage{hyperref}
\usepackage{cleveref}
\usepackage{algorithm}
\usepackage{algpseudocode}
\usepackage{algorithmicx}
\setlength{\parskip}{0.5em}
\setlength{\parindent}{0pt}
\theoremstyle{plain}
\theoremstyle{definition}
\newtheorem{definition}{Definition}
\newtheorem{identity}{Identity}
\newtheorem{conjecture}{Conjecture}
\newtheorem{theorem}{Theorem}
\newtheorem{lemma}{Lemma}
\newtheorem{proposition}{Proposition}
\newtheorem{example}{Example}
\crefname{definition}{Definition}{Definitions}
\crefname{identity}{Identity}{Identities}
\crefname{conjecture}{Conjecture}{Conjectures}
\crefname{theorem}{Theorem}{Theorems}
\crefname{lemma}{Lemma}{Lemmas}
\crefname{proposition}{Proposition}{Propositions}
\crefname{example}{Example}{Examples}
\usepackage{graphicx}
\usepackage{multicol}
\newcommand{\floor}[1]{\left\lfloor #1 \right\rfloor}
\newcommand{\totient}[1]{\phi\left({#1}\right)}
\newcommand{\bigO}[1]{O(#1)}
\newcommand{\softO}[1]{\tilde{O}(#1)}
\newcommand{\Z}{\mathbb{Z}}
\newcommand{\K}{K}
\newcommand{\abs}[1]{\text{abs}\left({#1}\right)}
\newcommand{\eval}[2]{\left . #1 \right|_{#2}}
\newcommand{\BT}{\textbf{B}}
\newcommand{\wt}[1]{\text{wt}(#1)}
\newcommand{\set}[1]{\left\langle #1 \right\rangle}

\begin{document}

\title{A Polynomial Ring Connecting Central Binomial Coefficients and Gould's Sequence}
\author{Joseph M. Shunia}
\date{October 2023 \\ \small Revised: December 2023 \normalsize}
\maketitle

\begin{abstract}
We establish a link between the central binomial coefficients $\binom{2n}{n}$ and Gould's sequence through the construction of multivariate polynomial quotient ring. We give a generalized definition for our specific ring structure, which is characterized by ideals that are generated from elements defined by specific polynomial recurrence relations. We explore a certain variation of this structure, and show that by expanding and evaluating  polynomials within the ring, it produces both $\binom{2n}{n}$ and Gould's sequence. Additionally, we describe a method for the calculation of binomial transforms for these famous sequences by leveraging the unique properties of our ring.
\end{abstract}

\section{Introduction}
The central binomial coefficients $\binom{2n}{n}$ \cite{A000984} and Gould's sequence \cite{A001316}, denoted $G_n$, are classical integer sequences. We uncover a new connection between these two fundamental combinatorial sequences by application of a specially designed multivariate polynomial quotient ring. We define a ring $\K_n = \mathbb{Z}[x_0, x_1, x_2, \ldots, x_n]/I$, where the ideal $I = \langle x_0^2 - P_0, x_1^2 - P_1, \ldots, x_n^2 - P_n \rangle$ is generated by the set $\{P_i = 2x_i + x_{i+1} \mid 0 \leq i \leq n \}$ in $\mathbb{Z}[x_0, x_1, x_2, \ldots, x_n]$. We demonstrate how expanding $(1 + x_0)^n$ in this ring yields the central binomial coefficients when evaluated at $x_0=x_1=x_2=\cdots=1$, and produces Gould's sequence when the coefficients are taken modulo $2$ prior to evaluation. This exposes an intriguing algebraic relationship between the sequences enabled by the structure of the polynomial ring.

While the specific sequences explored provide an example, the broader contribution is the ring structure itself. Constructing multivariate polynomial quotient rings with ideals that mimic recursive relations offers new means to calculate and manipulate nonlinear recursive sequences algebraically. The key insight is to design the polynomial quotient ring's ideals to directly mirror the recurrences that generate sequences of interest. Expanding polynomials within the ring then carries out the sequence generation process algebraically. This provides access to the tools of ring theory and polynomial manipulation, which may expose new sequence properties.

\section{Definitions}
\begin{definition}[Recursive quotient ring] \label{definition:recursivering}
Let \( R \) be a commutative ring with unity (e.g., \(\mathbb{Z}\), \(\mathbb{R}\), \(\mathbb{C}\), etc.). Consider the ring \( R[x_0, x_1, \ldots, x_n] \) consisting of polynomials in variables \( x_0, x_1, \ldots, x_n \) with coefficients in \( R \). Define \( I = \langle x_0^d - P_0, x_1^d - P_1, \ldots, x_n^d - P_n \rangle \) as an ideal of \( R[x_0, x_1, \ldots, x_n] \), where each \( P_i \) is a polynomial in \( R \) and takes the form:

\[ P_i = a_0 x_{i \cdot c_0 + j_0}^{k_0} + a_1 x_{i \cdot c_1 + j_1}^{k_1} + a_2 x_{i \cdot c_2 + j_2}^{k_2} + \cdots, \]

In this expression, \( a_m \in R \) are coefficients, \( c_m \) and \( j_m \) are integers where \( c_m \) acts as a scalar and \( j_m \) as a shift, and \( k_m \) are non-negative integers representing the exponents of the corresponding variables. The coefficients \( a_m \), scalars \( c_m \), shifts \( j_m \), and exponents \( k_m \) are fixed and do not depend on \( i \). The constant \( d \) can be an integer, a rational number, a complex number, etc., and does not necessarily match the nature of the coefficients in \( R \). The quotient ring \( R[x_0, x_1, \ldots, x_n]/I \) is defined as a recursive quotient ring if each variable \( x_i \) in the ring satisfies the recursive relation \( x_i^d = P_i \).
\end{definition}

\begin{definition}[Quotient ring $\K_n$] \label{definition:ring}
Let $\K_n = \mathbb{Z}[x_0, x_1, x_2, \ldots, x_n]/I$ be a recursive quotient ring (See \cref{definition:recursivering}) where the ideal \( I = \langle x_0^d - P_0, x_1^d - P_1, \ldots, x_n^d - P_n \rangle \) is generated by the set $\{P_i = 2x_i + x_{i+1} \mid 0 \leq i \leq n \}$ in $\mathbb{Z}[x_0, x_1, x_2, \ldots, x_n]$. In this ring, each variable $x_i$ satisfies the recursive relation $x_i^2 = 2x_i + x_{i+1}$.
\end{definition}

\begin{definition}[Modular quotient ring $(\K_n/m)$] \label{definition:modring}
Let $(\K_n/m) = (\mathbb{Z}/m)[x_0, x_1, x_2, \ldots, x_n]/I_m$, be a recursive quotient ring (See \cref{definition:recursivering}) where the ideal \( I = \langle x_0^d - P_0, x_1^d - P_1, \ldots, x_n^d - P_n \rangle \) is generated by the set $\{P_i = 2x_i + x_{i+1} \pmod{m} \mid 0 \leq i \leq n \}$ in $(\mathbb{Z}/m)[x_0, x_1, x_2, \ldots, x_n]$. In this ring, each variable $x_i$ satisfies the recursive relation $x_i^2 = 2x_i + x_{i+1} \pmod{m}$.
\end{definition}

\begin{definition}[Binomial transform function] \label{definition:binomialtransforms}
We define the function $\BT_t$, which takes in an integer sequence $a = \{ a_0, a_1, a_2, \ldots \}$, to be defined as the $t$-th binomial transform of the $a$ sequence terms, such that:
\begin{align*}
    \BT_{t}(a_n) =
    \begin{cases}
        a_n &\text{if } t = 0 \\
        \sum_{k=0}^{n} \binom{n}{k} \cdot a_k &\text{if } t = 1 \\
        \sum_{k=0}^{n} \binom{n}{k} \cdot (-1)^k \cdot a_k  &\text{if } t = -1 \\
        \sum_{k=0}^{n} \binom{n}{k} \cdot \BT_{t-1}(a_k) &\text{if } t > 1 \\
        \sum_{k=0}^{n} \binom{n}{k} \cdot (-1)^k \cdot \BT_{t+1}(a_k) &\text{if } t < -1 \\
    \end{cases}
\end{align*}
\end{definition}

\section[Connection to Central Binomial Coefficients]{Connection to $\binom{2n}{n}$}
\begin{theorem}
\label{theorem:cbc}
Let $n \in \Z^+$. Define the ring $\K_n$ as in \cref{definition:ring}. Evaluating the expansion of $(1+x_0)^n \in \K_n$ at $x_0=x_1=x_2=\cdots=1$ yields $\binom{2n}{n}$.
\end{theorem}
\begin{proof}
Consider the expression $(1+x_0)^n \in \K_n$. Upon expansion, the polynomial $(1+x_0)^n$ will contain powers of $x_0$, $x_0^2, x_0^3, \ldots, x_0^n$. Each power $a^k$ will be recursively replaced by polynomials with lower powers of $x_0$ and other variables $x_1, x_2, x_3, \ldots$. Specifically, we have:
\begin{align*}
    x_0^k = (2x_0+x_1)^{(k-1)} = \cdots = 2^k x_0 + \text{(terms involving $x_1, x_2, x_3, \ldots$)}.
\end{align*}
Substituting these into $(1+x_0)^n$, the coefficients for $x_0, x_1, x_2, \ldots$ essentially count the number of ways each $x_0$ in the initial $(1+x_0)^n$ is replaced by $x_1, x_2, x_3, \ldots$. When evaluated at $x_0=x_1=x_2=\cdots=1$, the expanded polynomial $(1+x_0)^n$ yields $\binom{2n}{n}$ since the coefficients are combinatorial in nature and count the number of ways to choose $n$ from $2n$.
\end{proof}

\section{Connection to Gould's Sequence}

Gould's sequence, entry A001316 in the OEIS \cite{A001316}, is an integer sequence that is connected to the binary expansion of integers, the central binomial coefficients, and Pascal's triangle.

To obtain the $n$-th term in Gould's sequence, which we will denote as $G_n$, we must first look at the binary representation of $n$. Counting the number of $1$s in the binary expansion of $n$ tells us its Hamming weight, which is often denoted as $\wt{n}$ \cite{Lin2004}. The $n$-th term in Gould's sequence is given by \cite{A001316}:
\begin{align}
    G_n = 2^\text{\wt{n}}
\end{align}
$G_n$ is connected to $\binom{2n}{n}$ in that it is the largest power of $2$ which divides $\binom{2n}{n}$. This result can be proven by induction using Kummer's Theorem \cite{Kummer1857}. $G_n$ also counts the number of odd terms in the $n$-th row of Pascal's triangle \cite{Glaisher1899}. That is, the number of odd terms in the polynomial expansion of $ (1+x)^n $.

Starting from $n=0$, Gould's sequence begins:
\begin{align}
    G_n = 1, 2, 2, 4, 2, 4, 4, 8, 2, 4, 4, 8, 4, 8, 8, 16, 2, 4, 4, 8, 4, \ldots
\end{align}

\begin{theorem}
\label{theorem:2}
Let $n \in \Z^+$. Define the ring $(\K_n/m)$ as in \cref{definition:modring} with $m=2$. Evaluating the expansion of $(1+x_0)^n \in (\K_n/2)$ at $x_0=x_1=x_2=\cdots=1$ yields the $n$-th term of Gould's sequence $G_n$, where $G_n = 2^{\wt{n}}$ and $\wt{n}$ is the Hamming weight of $n$.
\end{theorem}

\begin{proof}
Let us start by examining $ (1+x)^n \pmod{2} \in \Z[x]$. The coefficients of $(1+x)^n$ are given by the binomial coefficients $\binom{n}{k}$, and after taking the polynomial's coefficients modulo $2$, terms with even coefficients are eliminated, leaving only the terms with odd binomial coefficients.

Now, let us consider the polynomial $(1+x_0)^2 \in (\K_n/2)$ as an example. Expanding this using the substitution rules we get $1 + 2x_0 + x_0^2 = 1 + 2x_0 + (2x_0 + x_1)$, which is then taken modulo $2$ to get $1 + x_1$. Evaluating at $x_0=x_1=x_2=\cdots=1$ gives $2$, i.e., $x_0^2 \bmod 2 = 2^1$. This mirrors Gould's sequence for $n = 2$, $G_2 = 2^1$.

Extending this to $(1+x_0)^3 \pmod{2} \in (\K_n/2)$, the expansion begins as $1 + 3x_0 + 3x_0^2 + x_0^3 = 1 + 3x_0 + 3(1+x_1) + x_0(2x_0+x_1)$, which is then taken modulo $2$ to get $1 + x_0 + x_1 + x_0 x_1$. Evaluating at $x_0=x_1=x_2=\cdots=1$ gives $4$, i.e., $x_0^3 \bmod 2 = 2^2$, which also mirrors $G_3 = 2^2$.

Every time we consider a new power $n$, the terms that survive the modulo $2$ operation and evaluation at $x_0=x_1=x_2=\cdots=1$ essentially count the number of $1$'s in the binary representation of $n$, which is the Hamming weight $\wt{n}$. The surviving terms contribute to $2^{\wt{n}}$, the $n$-th term of Gould's sequence $G_n$.

Therefore, by evaluating the coefficients of $(1+x_0)^n \in (\K_n/2)$ at $x_0=x_1=x_2=\cdots=1$, we obtain $G_n = 2^{\wt{n}}$, thus proving the theorem.
\end{proof}

\section{Demonstrations}

\subsection{Central Binomial Coefficients} \label{section:demonstrations:cbc}
As an initial demonstration, we show how expanding the polynomial $(1+x_0)^n \in \K_n$, where $n \in \Z^+$ and $\K_n$ is defined by \cref{definition:ring}, generates polynomials which produce the central binomial coefficients $ \binom{2n}{n} $ when evaluated at $x_0=x_1=x_2=\cdots=1$:
\small
\begin{align*}
& \textbf{Polynomial} & \textbf{Evaluation}\\ 
& (1+x_0)^0 = 1 & 1 \\
& (1+x_0)^1 = 1+x_0 & 2 \\
& (1+x_0)^2 = 1+4x_0+x_1 & 6 \\
& (1+x_0)^3 = 1+13x_0+5x_1+x_0x_1 & 20 \\
& (1+x_0)^4 = 1+40x_0+20x_1+8x_0x_1+x_2 & 70 \\
& (1+x_0)^5 = 1+121x_0+76x_1+44x_0x_1+9x_2+x_0x_2 & 252 \\
& (1+x_0)^6 = 1+364x_0+285x_1+208x_0x_1+53x_2+12x_0x_2+x_1x_2 & 924 \\
& (1+x_0)^7 = 1+1093x_0+1065x_1+909x_0x_1+261x_2+89x_0x_2+13x_1x_2+x_0x_1x_2 & 3432 \\
& (1+x_0)^8 = 1+3280x_0+3976x_1+3792x_0x_1+1172x_2+528x_0x_2+104x_1x_2+16x_0x_1x_2+x_3 & 12870 \\
& \vdots & \vdots
\end{align*}
\normalsize

\subsection{Gould's Sequence}
Taking the coefficients of polynomials from \S\ref{section:demonstrations:cbc} modulo $2$, and then evaluating at $x_0=x_1=x_2=\cdots=1$, yields the terms of Gould's sequence. That is, we have $(1+x_0)^n \in (\K_n/2) $ where where $n \in \Z^+$ and $(\K_n/2)$ is defined by \cref{definition:modring}:

\small
\begin{align*}
& \textbf{Polynomial} & \textbf{Evaluation}\\ 
& (1+x_0)^0 = 1 & 1 \\
& (1+x_0)^1 = 1+x_0 & 2 \\
& (1+x_0)^2 = 1+x_1 & 2 \\
& (1+x_0)^3 = 1+x_0+x_1+x_0x_1 & 4 \\
& (1+x_0)^4 = 1+x_2 & 2 \\
& (1+x_0)^5 = 1+x_0+x_2+x_0x_2 & 4 \\
& (1+x_0)^6 = 1+x_1+x_2+x_1x_2 & 4 \\
& (1+x_0)^7 = 1+x_0+x_1+x_0x_1+x_2+x_0x_2+x_1x_2+x_0x_1x_2 & 8 \\
& (1+x_0)^8 = 1+x_3 & 2 \\
\vdots & & \vdots
\end{align*}
\normalsize

\section{Binomial Transforms}
A useful feature of recursive quotient rings we've defined (See \cref{definition:recursivering}) is that they can be used to calculate the binomial transforms of the sequences they generate. To this end, we offer a brief demonstration.

\begin{proposition} \label{proposition:binomialtransforms}
Let $n \in \Z^+$. Define the ring $\K_n$ as in \cref{definition:ring}. Let $\BT_t$ be the binomial transform function defined in \cref{definition:binomialtransforms}. Let $a$ be the integer sequence $a_n=\binom{2n}{n}$. Evaluating the expansion of $(t + 1 + x_0)^n \in \K_n$ at $x_0=x_1=x_2=\cdots=1$ equals $\BT_t(a_n)$, the $t$-th the binomial transform of the central binomial coefficients.
\end{proposition}
\begin{proof}
Let $P$ be a polynomial in $\K_n$. $P := 1 + x_0$. By the binomial theorem:
\begin{align*}
    (1 + P)^n = \sum_{k=0}^{n} \binom{n}{k} P^k
\end{align*}
Evaluating this at $x_0=x_1=x_2=\cdots=1$ yields the binomial transform of the sequence generated by $P^k = (1 + x_0)^k \in \K_n$ for each $k$ in the sum, whose valuation we know to be $\binom{2k}{k}$ (by \cref{theorem:cbc}). This gives us the binomial transform for $t=1$. Hence, if we shift by some integer $t$ instead of $1$, we compute the $t$-th binomial transform. This result follows directly from the binomial theorem and how it applies to integer powers.
\end{proof}

Taking a similar approach to \cref{proposition:binomialtransforms} given above, one can also compute the $t$-th binomial transform of Gould's sequence $G_n$. Calculating the binomial transforms of $G_n$ requires a subtle alteration to our quotient ring $K_n$. That is, we must adjust the ideal to mimic the behavior of taking the coefficients modulo $2$, but without restricting the polynomial coefficients to $(\Z/2)$. We can achieve this by changing the polynomial recurrence which generates the ideal to instead follow the recursive pattern $P_{i}^2 = -2 x_i + x_{i+1}$. In this new ring, each variable $x_i$ satisfies the recursive relation $x_i^2 = -2x_i + x_{i+1}$. When expanding $(1+x_0)^n$ within the modified ring, the $-2x_i$ terms will cause the even terms to cancel out, which mimics the behavior of taking the coefficients modulo $2$.

\begingroup
\raggedright
\bibliographystyle{unsrtnat}
\bibliography{main}
\endgroup

\end{document}