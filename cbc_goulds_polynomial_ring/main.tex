\pdfoutput=1
\documentclass{article}
\usepackage{fullpage}
\usepackage{amsmath, amssymb, amsthm}
\usepackage[utf8]{inputenc}
\usepackage[english]{babel}
\usepackage[numbers]{natbib}
\usepackage{csquotes}
\usepackage{url}
\usepackage{hyperref}
\usepackage{cleveref}
\usepackage{algorithm}
\usepackage{algpseudocode}
\usepackage{algorithmicx}
\setlength{\parskip}{0.5em}
\setlength{\parindent}{0pt}
\theoremstyle{plain}
\theoremstyle{definition}
\newtheorem{definition}{Definition}
\newtheorem{notation}{Notation}
\newtheorem{identity}{Identity}
\newtheorem{conjecture}{Conjecture}
\newtheorem{theorem}{Theorem}
\newtheorem{lemma}{Lemma}
\newtheorem{proposition}{Proposition}
\newtheorem{example}{Example}
\crefname{definition}{Definition}{Definitions}
\crefname{notation}{Notation}{Notations}
\crefname{identity}{Identity}{Identities}
\crefname{conjecture}{Conjecture}{Conjectures}
\crefname{theorem}{Theorem}{Theorems}
\crefname{lemma}{Lemma}{Lemmas}
\crefname{proposition}{Proposition}{Propositions}
\crefname{example}{Example}{Examples}
\usepackage{graphicx}
\usepackage{multicol}
\newcommand{\floor}[1]{\left\lfloor #1 \right\rfloor}
\newcommand{\totient}[1]{\phi\left({#1}\right)}
\newcommand{\bigO}[1]{O(#1)}
\newcommand{\softO}[1]{\tilde{O}(#1)}
\newcommand{\Z}{\mathbb{Z}}
\newcommand{\K}{K}
\newcommand{\abs}[1]{\text{abs}\left({#1}\right)}
\newcommand{\eval}[2]{\left . #1 \right|_{#2}}
\newcommand{\BT}{\textbf{B}}
\newcommand{\wt}[1]{\#(#1)}
\newcommand{\set}[1]{\left\langle #1 \right\rangle}

\begin{document}

\title{A Polynomial Ring Connecting Central Binomial Coefficients and Gould's Sequence}
\author{Joseph M. Shunia}
\date{October 2023 \\ \small Revised: December 2023 \normalsize}
\maketitle

\begin{abstract}
We establish a link between the central binomial coefficients $\binom{2n}{n}$ and Gould's sequence through the construction of a multivariate polynomial quotient ring. We give a generalized definition for our specific ring structure, which is characterized by ideals that are generated from elements defined by a kind of polynomial recurrence relation. We explore a certain variation of this structure, and show that by expanding and evaluating  polynomials within the ring, it produces both $\binom{2n}{n}$ and Gould's sequence. Additionally, we describe a method for the calculation of binomial transforms for these famous sequences by leveraging the unique properties of our ring.
\end{abstract}

\section{Introduction}
The central binomial coefficients $\binom{2n}{n}$ \cite{A000984} and Gould's sequence \cite{A001316}, denoted $G$, are classical integer sequences. We uncover a new connection between these two fundamental combinatorial sequences by application of a specially designed multivariate polynomial quotient ring. We define a ring $\K_n = \Z[x_1, x_2, \ldots, x_n]/I$, where the ideal $I = \langle x_1^2 - P_1, x_2^2 - P_2, \ldots, x_n^2 - P_n \rangle$ is generated by the polynomial recurrence $P_i = 2x_i + x_{i+1}, \text{ } \forall i : 1 \leq i < n$. We demonstrate how expanding $(1 + x_1)^n$ in this ring yields the central binomial coefficients when evaluated at $x_1=x_2=\cdots=x_n=1$, and produces Gould's sequence when the coefficients are taken modulo $2$ prior to evaluation. This exposes an intriguing algebraic relationship between the sequences enabled by the structure of the polynomial ring.

While the specific sequences explored provide an example, the broader contribution is the ring structure itself. Constructing multivariate polynomial quotient rings with ideals that mimic recursive relations offers new means to calculate and manipulate nonlinear recursive sequences algebraically. The key insight is to design the polynomial quotient ring's ideals to follow the recurrences that generate sequences of interest. Expanding polynomials within the ring then carries out the sequence generation process algebraically. This provides access to the tools of ring theory and polynomial manipulation, which may expose new sequence properties.

\section{Definitions}
\subsection{General}

\begin{definition}[Recursive quotient ring] \label{definition:recursivering}
Let \( R \) be a commutative ring with unity (e.g., \(\Z\), \(\mathbb{R}\), \(\mathbb{C}\), etc.).

Consider the ring \( S = R[x_1, x_2, \ldots, x_{n} ] \) consisting of polynomials in variables \( x_1, x_2, \ldots, x_n \) with coefficients in \( R \).

Define \( I = \langle x_1^d - P_1, x_2^d - P_2, \ldots, x_n^d - P_n \rangle \) as an ideal of \( S \), where each \( P_i \) is a polynomial in \( S \) and takes the form:
\begin{align}
    P_i = a_0 + a_1 x_{i \cdot c_1 + j_1}^{k_1} + a_2 x_{i \cdot c_2 + j_2}^{k_2} + \cdots
\end{align}

In this expression, the \( a_m \) are coefficients in \( R \) and/or polynomials in \( S \) (defined by recurrence or otherwise). The \( c_m \) and \( j_m \) are integers where \( c_m \) acts as a scalar and \( j_m \) as a shift, and \( k_m \) represent the exponents of the corresponding variables. The scalars \( c_m \), shifts \( j_m \), and exponents \( k_m \) are fixed and do not depend on \( i \). The constant \( d \) and the exponents \( k_m \) can be in \(\Z\), \(\mathbb{R}\), \(\mathbb{C}\), etc., and do not necessarily match the domain of the coefficients in \( S \).

The quotient ring \( S = R[x_1, x_2, \ldots, x_{n}]/I \) is defined as a \textbf{recursive quotient ring} if and only if for all \( x_i \) in \( K \), the relation \( x_i^d = P_i \) is satisfied, and the polynomials \( P_i \) are generated recursively for all \( i \) in the range \( \alpha \leq i \leq \omega \), where \( \alpha \) and \( \omega \) are specified start and end indices. For all indices \( i \) not in this range, \( x_i^d \) is assumed to be zero within the ring \( S \) unless it is explicitly stated otherwise.
\end{definition}

\begin{definition}[Binomial transform function] \label{definition:binomialtransforms}
We define the operator $\BT_t$, which takes in an integer sequence $a = \{ a_0, a_1, a_2, \ldots \}$, to be defined as the $t$-th binomial transform of the $a$ sequence terms, such that:
\begin{align*}
    \BT_{t}(a_n) =
    \begin{cases}
        a_n &\text{if } t = 0 \\
        \sum_{k=0}^{n} \binom{n}{k} \cdot a_k &\text{if } t = 1 \\
        \sum_{k=0}^{n} \binom{n}{k} \cdot \BT_{t-1}(a_k) &\text{if } t > 1 \\
        \sum_{k=0}^{n} \binom{n}{k} \cdot (-1)^{n-k} \cdot a_k  &\text{if } t = -1 \\
        \sum_{k=0}^{n} \binom{n}{k} \cdot (-1)^{n-k} \cdot \BT_{t+1}(a_k) &\text{if } t < -1 \\
    \end{cases}
\end{align*}
\end{definition}

\subsection{Applied}

\begin{definition}[Quotient ring $\K_n$] \label{definition:ring}
Let $\K_n = \Z[x_1, x_2, \ldots, x_n]/I$ be a recursive quotient ring (See \cref{definition:recursivering}) with an ideal \( I = \langle x_1^d - P_1, x_2^d - P_2, \ldots, x_n^d - P_n \rangle \). The polynomials $P_i$ in the generators of $I$ are defined by the function: 
\begin{align}
P_i =
\begin{cases}
    2x_i + x_{i+1} & \text{ if } 1 \leq i < n \\
    0 &\text{ if } i = n
\end{cases}
\end{align}
In this ring, the variables $x_i$ satisfy the recursive relation $x_i^2 = 2x_i + x_{i+1}$ for $1 \leq i < n$, where $x_{i+1}$ refers to the next variable in the sequence, and $x_n^2 = 0$.
\end{definition}

\begin{definition}[Modular quotient ring $(\K_n/m)$] \label{definition:modring}
Let $(\K_n/m) = (\Z/m)[x_1, x_2, \ldots, x_n]/I$ be a recursive quotient ring (See \cref{definition:recursivering}) with an ideal \( I = \langle x_1^d - P_1, x_2^d - P_2, \ldots, x_n^d - P_n \rangle \) and coefficients in $(\Z/m)$. The polynomials $P_i$ in the generators of $I$ are defined by the function: 
\begin{align}
P_i =
\begin{cases}
    2x_i + x_{i+1} \pmod{m} & \text{ if } 1 \leq i < n \\
    0 \pmod{m} &\text{ if } i = n
\end{cases}
\end{align}

In this ring, the variables $x_i$ satisfy the recursive relation $x_i^2 = 2x_i + x_{i+1} \pmod{m}$ for $1 \leq i < n$, where $x_{i+1}$ refers to the next variable in the sequence, and $x_n^2 = 0 \pmod{m}$.
\end{definition}

\begin{definition}[Quotient ring $\K'_n$] \label{definition:ring2}
Let $\K'_n = \Z[x_1, x_2, \ldots, x_n]/I$ be a recursive quotient ring (See \cref{definition:recursivering}) with an ideal \( I = \langle x_1^d - P_1, x_2^d - P_2, \ldots, x_n^d - P_n \rangle \). The polynomials $P_i$ in the generators of $I$ are defined by the function: 
\begin{align}
P_i =
\begin{cases}
    -2x_i + x_{i+1} & \text{ if } 1 \leq i < n \\
    0 &\text{ if } i = n
\end{cases}
\end{align}
In this ring, the variables $x_i$ satisfy the recursive relation $x_i^2 = -2x_i + x_{i+1}$ for $1 \leq i < n$, where $x_{i+1}$ refers to the next variable in the sequence, and $x_n^2 = 0$.
\end{definition}

\section[Connection to Central Binomial Coefficients]{Connection to $\binom{2n}{n}$} \label{section:cbc}
\begin{theorem} \label{theorem:cbc}
\textit{
Let $n \in \Z^+$. Let $b={\floor{\log_2(n)}+2}$. Define the ring $\K_b$ as in \cref{definition:ring}. Evaluating the expansion of $(1+x_1)^n \in \K_b$ at $x_1=x_2=\cdots=x_n=1$ yields $\binom{2n}{n}$.
}
\end{theorem}
\begin{proof}
First, observe that by the process of exponentiation by squaring \cite{knuth1997art}, expanding $(1+x_1)^n \in \K_b$ requires at most $\log_2(n)$ squarings. Hence, $b={\floor{\log_2(n)}+2}$ is sufficient to cover all of the necessary variables when expanding $(1+x_1)^n \in \K_b$.

Consider the expression $(1+x_1)^n \in \K_b$. The binomial expansion of this polynomial yields terms of the form $\binom{n}{k} x_1^k$, for $k$ ranging from $0$ to $n$. In $\K_b$, the recursive relation $x_i^2 = 2x_i + x_{i+1}$ modifies the expansion by replacing each instance of $x_i^2$ with $2x_i + x_{i+1}$.

Upon expansion, the polynomial $(1+x_1)^n$ will contain powers of $x_1$, $x_1^2, x_1^3, \ldots, x_1^n$. Each power $x_1^k$ will be recursively replaced by polynomials with lower powers of $x_1$ and other variables $x_2, x_3, x_4, \ldots$. Specifically, we have:
\begin{align}
    x_1^k = (2x_1+x_2)^{(k-1)} = \cdots = 2^k x_1 + \text{(terms involving $x_2, x_3, x_4, \ldots$)}
\end{align}

Substituting these into $(1+x_1)^n$, the coefficients for $x_1, x_2, x_3, \ldots$ essentially count the number of ways each $x_1$ in the initial $(1+x_1)^n$ is replaced by $x_2, x_3, x_4, \ldots$. When evaluated at $x_1=x_2=\cdots=x_n=1$, the expanded polynomial $(1+x_1)^n$ yields $\binom{2n}{n}$ since the coefficients are combinatorial in nature and count the number of ways to choose $n$ from $2n$.
\end{proof}

\section{Connection to Gould's Sequence} \label{section:goulds}

Gould's sequence, entry A001316 in the OEIS \cite{A001316}, is an integer sequence that is connected to the binary expansion of integers, the central binomial coefficients, and Pascal's triangle. It is named after the mathematician Henry Gould \cite{A001316}.

To obtain the $n$-th term in Gould's sequence, which we will denote as $G_n$, we must first look at the binary representation of $n$. Counting the number of $1$s in the binary expansion of $n$ tells us its Hamming weight, which is often denoted as $\wt{n}$ \cite{Lin2004}. The $n$-th term in Gould's sequence $G_n$ is given by \cite{A001316}:
\begin{align}
    G_n = 2^{\wt{n}}
\end{align}
$G_n$ is connected to $\binom{2n}{n}$ in that it is the largest power of $2$ which divides $\binom{2n}{n}$. This result follows from Kummer's Theorem \cite{Kummer1857}. $G_n$ also counts the number of odd terms in the $n$-th row of Pascal's triangle \cite{Glaisher1899}. That is, the number of odd terms in the polynomial expansion of $(1+x)^n \in \Z[x]$.

Starting from $n=0$, Gould's sequence begins:
\begin{align*}
    G = \{ 1, 2, 2, 4, 2, 4, 4, 8, 2, 4, 4, 8, 4, 8, 8, 16, 2, 4, 4, 8, 4, 8, 8, 16, 4, 8, 8, 16, 8, 16, \ldots \}
\end{align*}

\begin{theorem}
\label{theorem:goulds}
\textit{
Let $n \in \Z^+$. Let $b={\floor{\log_2(n)}+2}$. Define the ring $(\K_b/m)$ as in \cref{definition:modring} with $m=2$. Expanding $(1+x_1)^n \in (\K_b/2)$ and then evaluating in $\Z$ at $x_1=x_2=\cdots=x_n=1$ yields the $n$-th term of Gould's sequence $G_n$. Where $G_n = 2^{\wt{n}}$ and $\wt{n}$ is the Hamming weight of $n$.
}
\end{theorem}
\begin{proof}
First, observe that by the process of exponentiation by squaring \cite{knuth1997art}, expanding $(1+x_1)^n \in (\K_b/2)$ requires at most $\log_2(n)$ squarings. Hence, $b={\floor{\log_2(n)}+2}$ is sufficient to cover all of the necessary variables when expanding $(1+x_1)^n \in (\K_b/2)$.

Next, we proceed by induction on $n$ to show that the expanded polynomial yields $G_n = 2^{\wt{n}}$ upon evaluation in $\Z$ at $x_1 = x_2 = x_3 = \cdots = 1$.

Consider $n = 1$. In this case, $(1 + x_1)^1 = 1 + x_1 \in (\K_b/2)$. Evaluating in $\Z$ at $x_1 = x_2 = x_3 = \cdots = 1$ yields $2$, which is $2^{\wt{1}} = 2^1$. Thus, the statement holds for $n = 1$.

Assume the statement holds for some $k \geq 1$, that is, expanding $(1 + x_1)^k \in (\K_b/2)$ and then evaluating in $\Z$ at $x_1 = x_2 = x_3 = \cdots = 1$ yields $2^{\wt{k}}$. We need to show it holds for $k+1$.

Consider $(1 + x_1)^{k+1}$ in $(\K_b/2)$. This can be written as $(1 + x_1)^k (1 + x_1)$. Using the inductive hypothesis, we know that $(1 + x_1)^k$ yields $2^{\wt{k}}$ when evaluated in $\Z$. Now, we need to consider the additional factor $(1 + x_1)$.

In the ring $(\K_b/2)$, the expansion of $(1 + x_1)^{k+1}$ will result in various terms involving $x_1, x_2, \ldots$, with each term corresponding to a particular combination of bits in the binary representation of $k+1$. The modulo $2$ operation ensures that only terms corresponding to odd counts of $x_1$ will contribute to the final sum. That is, positions with $1$ in the binary representation of $k+1$.

When we evaluate this expression in $\Z$ at $x_1 = x_2 = x_3 = \cdots = 1$, the result will be a power of $2$, specifically $2^{\wt{k+1}}$, where $\wt{k+1}$ is the Hamming weight of $k+1$. This is because the terms that survive in the expansion and modulo $2$ reduction directly correspond to the count of $1$s in the binary representation of $k+1$.

Therefore, by induction, expanding $(1 + x_1)^n \in (\K_b/2)$ and then evaluating in $\Z$ at $ x_1 = x_2 = x_3 = \cdots = 1$ yields $2^{\wt{n}}$ for all $n \in \Z^+$.
\end{proof}

\section{Demonstrations}

\subsection{Central Binomial Coefficients} \label{section:demonstrations:cbc}
Let $n \in \Z^+$, $b={\floor{\log_2(n)}+2}$, and $\K_b$ be the ring defined by \cref{definition:ring}. We will demonstrate how expanding the polynomial $(1+x_1)^n \in \K_b$, generates polynomials which produce the central binomial coefficients $\binom{2n}{n}$ when evaluated at $x_1=x_2=\cdots=x_n=1$:
\small
\begin{align*}
& \textbf{Polynomial} \text{ in } \K_b & \textbf{Evaluation} \text{ in } \K_b \\ 
& (1+x_1)^0 = 1 & 1 \\
& (1+x_1)^1 = 1+x_1 & 2 \\
& (1+x_1)^2 = 1+4x_1+x_2 & 6 \\
& (1+x_1)^3 = 1+13x_1+5x_2+x_1x_2 & 20 \\
& (1+x_1)^4 = 1+40x_1+20x_2+8x_1x_2+x_3 & 70 \\
& (1+x_1)^5 = 1+121x_1+76x_2+44x_1x_2+9x_3+x_1x_3 & 252 \\
& (1+x_1)^6 = 1+364x_1+285x_2+208x_1x_2+53x_3+12x_1x_3+x_2x_3 & 924 \\
& (1+x_1)^7 = 1+1093x_1+1065x_2+909x_1x_2+261x_3+89x_1x_3+13x_2x_3+x_1x_2x_3 & 3432 \\
& (1+x_1)^8 = 1+3280x_1+3976x_2+3792x_1x_2+1172x_3+528x_1x_3+104x_2x_3+16x_1x_2x_3+x_4 & 12870 \\
& \vdots & \vdots
\end{align*}
\normalsize

\subsection{Gould's Sequence} \label{section:demonstrations:goulds}
Let $(1+x_1)^n \in (\K_b/2)$ where where $n \in \Z^+$, $b={\floor{\log_2(n)}+2}$, and $(\K_b/2)$ is defined by \cref{definition:modring}. We will demonstrate how taking the coefficients of polynomials from \S\ref{section:demonstrations:cbc} modulo $2$, and then evaluating in $\Z$ at $x_1=x_2=\cdots=x_n=1$ yields the $n$-th term of Gould's sequence $G_n$:

\small
\begin{align*}
& \textbf{Polynomial} \text{ in } (\K_b/2) & \textbf{Evaluation} \text{ in } \Z \\ 
& (1+x_1)^0 = 1 & 1 \\
& (1+x_1)^1 = 1+x_1 & 2 \\
& (1+x_1)^2 = 1+x_2 & 2 \\
& (1+x_1)^3 = 1+x_1+x_2+x_1x_2 & 4 \\
& (1+x_1)^4 = 1+x_3 & 2 \\
& (1+x_1)^5 = 1+x_1+x_3+x_1x_3 & 4 \\
& (1+x_1)^6 = 1+x_2+x_3+x_2x_3 & 4 \\
& (1+x_1)^7 = 1+x_1+x_2+x_1x_2+x_3+x_1x_3+x_2x_3+x_1x_2x_3 & 8 \\
& (1+x_1)^8 = 1+x_4 & 2 \\
\vdots & & \vdots
\end{align*}
\normalsize

\section{Binomial Transforms}
A useful feature of recursive quotient rings we've defined (See \cref{definition:recursivering}) is that they can be used to calculate the binomial transforms of the sequences they generate. To this end, we offer a brief demonstration.

\subsection{Transforming the Central Binomial Coefficients}
\begin{proposition} \label{proposition:binomialtransforms}
\textit{
Let $n \in \Z^+$. Let $b={\floor{\log_2(n)}+2}$. Define the ring $\K_b$ as in \cref{definition:ring}. Let $\BT_t$ be the binomial transform function defined in \cref{definition:binomialtransforms}. Let $a$ be the integer sequence $a_n=\binom{2n}{n}$. Evaluating the expansion of $(t + 1 + x_1)^n \in \K_b$ at $x_1=x_2=\cdots=x_n=1$ equals $\BT_t(a_n)$, the $t$-th binomial transform of the central binomial coefficients from $0$ to $n$.
}
\end{proposition}
\begin{proof}
Let $P$ be a polynomial in $\K_b$. $P := 1 + x_1$. By the binomial theorem:
\begin{align*}
    (1 + P)^n = \sum_{k=0}^{n} \binom{n}{k} P^k \in \K_b
\end{align*}
Evaluating this at $x_1=x_2=\cdots=x_n=1$ yields the binomial transform of the sequence generated by $P^k = (1 + x_1)^k \in \K_b$ for each $k$ in the sum, whose valuation we know to be $\binom{2k}{k}$ (by \cref{theorem:cbc}). This gives us the binomial transform for $t=1$. Hence, if we shift by some integer $t$ instead of $1$, we compute the $t$-th binomial transform. This result follows directly from the binomial theorem and how it applies to integer powers.
\end{proof}

\subsection{Transforming Gould's Sequence}
Using a similar approach to \cref{proposition:binomialtransforms} given above, we can compute the $t$-th binomial transform of Gould's sequence. However, calculating the binomial transforms of Gould's sequence requires a different approach to calculating $G_n$ than the approach used in our quotient ring $(K_m/2)$ (See \cref{definition:modring}). Specifically, we must define an ideal which mimics the behavior of taking the coefficients modulo $2$, but without restricting the polynomial coefficients to $(\Z/2)$. Otherwise, the binomial transform will be taken modulo $2$.

\begin{theorem} \label{theorem:goulds2}
\textit{
Let $n \in \Z^+$. Let $b={\floor{\log_2(n)}+2}$. Define the ring $\K'_b$ as in \cref{definition:ring2}. Expanding $(1+x_1)^n \in \K'_b$ and then evaluating at $x_1=x_2=\cdots=x_n=1$ yields the $n$-th term of Gould's sequence $G_n$. Where $G_n = 2^{\wt{n}}$ and $\wt{n}$ is the Hamming weight of $n$.
}
\end{theorem}
\begin{proof}
In \cref{theorem:goulds}, we showed how expanding $(1+x_1)^n \in (\K_b/2)$ and then evaluating in $\Z$ at $x_1=x_2=\cdots=x_n=1$ yields $G_n$.

The proof of \cref{theorem:goulds} does not obviously apply, as in the ring $\K'_b$, we are not taking coefficients modulo $2$. Instead, we have constructed a ring similar to $K_b$ as defined in \cref{definition:ring}, however, we have changed the polynomial recurrence which generates the ideal to follow the recursive pattern $P_{i}^2 = -2 x_i + x_{i+1}$. This implies that each variable $x_i$ satisfies the recursive relation $x_i^2 = -2x_i + x_{i+1}$.

When expanding $(1+x_1)^n \in \K'_b$, the $-2x_i$ terms will cause the terms with even coefficients to cancel out, and will leave a remainder of $1$ for all of the odd terms after subtracting. This exactly mimics the behavior of taking the coefficients modulo $2$. Hence, by \cref{theorem:goulds}, expanding $(1+x_1)^n \in \K'_b$ and then evaluating at $x_1=x_2=\cdots=x_n=1$ yields $G_n$.
\end{proof}

Finally, we will conclude by showing how the ring $K'_n$ can be used to calculate binomial transforms of Gould's sequence.

\begin{proposition} \label{proposition:gouldbinomialtransforms}
\textit{
Let $n \in \Z^+$. Let $b={\floor{\log_2(n)}+2}$. Define the ring $\K'_b$ as in \cref{definition:ring2}. Let $\BT_t$ be the binomial transform function defined in \cref{definition:binomialtransforms}. Let $G_n$ represent the $n$-th term of Gould's sequence. Evaluating the expansion of $(t + 1 + x_1)^n \in \K'_b$ at $x_1=x_2=\cdots=x_n=1$ equals $\BT_t(G_n)$, the $t$-th binomial transform of Gould's sequence $G$ from $0$ to $n$.
}
\end{proposition}
\begin{proof}
Let $P$ be a polynomial in $\K'_b$. $P := 1 + x_1$. By the binomial theorem:
\begin{align*}
    (1 + P)^n = \sum_{k=0}^{n} \binom{n}{k} P^k \in \K'_b
\end{align*}
Evaluating this at $x_1=x_2=\cdots=x_n=1$ yields the binomial transform of the sequence generated by $P^k = (1 + x_1)^k \in \K'_b$ for each $k$ in the sum, whose valuation we know to be $G_n$ (by \cref{theorem:goulds2}). This gives us the binomial transform for $t=1$. Hence, if we shift by some integer $t$ instead of $1$, we compute the $t$-th binomial transform. This result follows directly from the binomial theorem and how it applies to integer powers.
\end{proof}

\subsection{Terms in the Binomial Transform of Gould's Sequence}
The binomial transform of Gould's sequence is entry A368655 in the OEIS \cite{A368655}. Starting from $n=0$, Gould's sequence begins:

$\BT_{1}(G_n) = \{ 1, 3, 7, 17, 39, 85, 181, 387, 839, 1829, 3953, 8391, 17461, 35759, 72559, 146921, 298631, 611733, \ldots \}$

\textbf{Remark.}
The straightforward computation of the binomial transforms of Gould's sequence within our polynomial ring structure is quite intriguing. Gould's sequence does not follow a simple increasing trend; rather, its terms oscillate in a sawtooth-like fashion and are tied to the binary representation of integers (See \S\ref{section:goulds}). The sequence's terms, while being powers of $2$, are distributed throughout in a seemingly random order. In the binomial transform process, we multiply each of these irregularly spaced elements by binomial coefficients. The fact that this can be done smoothly in our polynomial ring setup, without individually calculating each term, is somewhat baffling. It appears that the binary essence of integers is subtly embedded in the exponentiation of polynomials within our ring. This finding, while not earth-shattering, is nonetheless unexpected given the inherent complexity of Gould's sequence. Further investigation may offer deeper insights into the behavior of Gould's sequence and similar seemingly chaotic integer sequences.

\begingroup
\raggedright
\bibliographystyle{unsrtnat}
\bibliography{main}
\endgroup

\end{document}