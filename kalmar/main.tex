\pdfoutput=1
\documentclass[10pt,a4paper]{article}
\usepackage[margin=1.5cm]{geometry}
\usepackage{amsmath, amssymb, amsthm}
\usepackage[utf8]{inputenc}
\usepackage[english]{babel}
\usepackage[numbers]{natbib}
\usepackage{url}
\usepackage[usenames]{color}
\usepackage[colorlinks=true,
linkcolor=webgreen,
filecolor=webbrown,
citecolor=webgreen]{hyperref}
\definecolor{webgreen}{rgb}{0,.5,0}
\definecolor{webbrown}{rgb}{.6,0,0}
\definecolor{red}{rgb}{1,0,0}
\usepackage{cleveref}
\usepackage{tabularx}
\theoremstyle{plain}
\newtheorem{proposition}{Proposition}[section]
\newtheorem{conjecture}{Conjecture}[section]
\newtheorem{corollary}{Corollary}[section]
\newtheorem{lemma}{Lemma}[section]
\newtheorem{definition}{Definition}[section]
\newtheorem{theorem}{Theorem}[section]
\newtheorem{remark}{Remark}[section]
\newtheorem{example}{Example}[section]
\newtheorem{question}{Question}[section]
\newtheorem{counterexample}{Counter Example}[section]
\newtheorem{observation}{Observation}[section]
\crefname{conjecture}{Conjecture}{Conjectures}
\crefname{theorem}{Theorem}{Theorems}
\crefname{corollary}{Corollary}{Corollaries}
\crefname{lemma}{Lemma}{Lemmas}
\crefname{proposition}{Proposition}{Propositions}
\crefname{remark}{Remark}{Remarks}
\crefname{definition}{Definition}{Definitions}
\crefname{notation}{Notation}{Notations}
\crefname{example}{Example}{Examples}
\crefname{question}{Question}{Questions}
\crefname{section}{\S}{Sections}
\newcommand{\abs}[1]{\left| #1 \right|}
\newcommand{\stirling}[2]{\left[ #1 \atop #2 \right]}
\newcommand{\floor}[1]{\left\lfloor #1 \right\rfloor}
\newcommand{\ceil}[1]{\left\lceil #1 \right\rceil}
\newcommand{\round}[1]{\left[ #1 \right]}
\newcommand{\numerator}{\mathrm{numerator}}
\newcommand{\denominator}{\mathrm{denominator}}
\newcommand{\sg}{\mathrm{sg}}
\newcommand{\xor}{\oplus}
\newcommand{\hw}{\operatorname{HW}}
\newcommand{\lpf}{\operatorname{LPF}}
\newcommand{\carry}{\operatorname{\Delta_{+}}}
\newcommand{\lcm}{\mathrm{lcm}}
\newcommand{\mask}{\mathrm{mask}}
\newcommand{\coprime}{\epsilon}
\newcommand{\rmod}{\operatorname{rmod}}
\newcommand{\Z}{\mathbb{Z}}
\newcommand{\K}{\mathcal{K}}
\newcommand{\wt}[1]{\#(#1)}
\newcommand{\eval}[2]{\left . #1 \right|_{#2}}
\newcommand{\seqnum}[1]{\href{https://oeis.org/#1}{\rm \underline{#1}}}
\newcommand{\ahref}[2]{\href{#1}{\rm #2}}
\setlength{\parskip}{0.5em}
\setlength{\parindent}{0pt}

\title{[Draft] On Kalmar Numbers and Arithmetic Terms}
\author{Joseph M. Shunia}
\date{August 13, 2024}

\begin{document}

\maketitle

\begin{abstract}
\noindent From the classes of Kalmar functions and arithmetic terms, we describe the class of Kalmar numbers, which is the subset of computable real numbers that can be represented as a limit of the quotient of two Kalmar functions. We give Kalmar numbers for many important mathematical constants and functions, such as $e$, $\pi$, $\sqrt[r]{n}$, $\log(n)$, $\exp(n)$, $\sin(n)$, $\cos(n)$, $\Gamma_k(q)$, $(q)_{n,k}$, $\psi^{(k)}(q)$.
\end{abstract}

\textbf{DISCLAIMER:} This paper is a work in progress. Many proofs and results are currently missing.

\section{Kalmar Numbers}
We define a \textbf{Kalmar number} as a computable real number which can be represented as a limit of the quotient of two Kalmar functions. That is, $r \in \mathbb{R}$ is a Kalmar number if there exists a limit
\begin{align*}
r = \lim_{n \to +\infty} \frac{f(n)}{g(n)}
\quad \text{ or } \quad 
r = \lim_{n \to -\infty} \frac{f(n)}{g(n)}
\quad \text{ or } \quad 
r = \lim_{n \to 0} \frac{f(n)}{g(n)}
\end{align*}
where $f(n),g(n)$ are Kalmar functions. This definition is due to Lorenzo Sauras-Altuzarra, who described it in a private correspondence.

\section{Characterizing the Set of Kalmar Numbers}
An interesting question from Mihai Prunescu is:
\begin{question}
What is the subset of computable real numbers that are Kalmar numbers?
\end{question}
We offer a partial solution to the question.

\begin{theorem}
Let $r \in \mathbb{R}$ such that the ratios of consecutive terms for both the numerators and denominators of the $n$-th convergents in its generalized continued fraction representation are bounded by a tower of exponentials $2^{2^{\ldots^{2^n}}}$ with height $n$. Then, there exist Kalmar functions $f(n)$ and $g(n)$ such that
\begin{align*}
r = \lim_{n\to\infty} \frac{f(n)}{g(n)} .
\end{align*}
\end{theorem}
\begin{proof}
The generalized continued fraction representation of $r$ is given by
\begin{align*}
r = a_0 + \cfrac{b_1}{a_1 + \cfrac{b_2}{a_2 + \cfrac{b_3}{a_3 + \ddots}}} ,
\end{align*}
where $a_i$ and $b_i$ are sequences of integers. The $n$-th convergents of the generalized continued fraction for $r$ can be expressed using the recurrence relations
\begin{align*}
P_n &= a_n P_{n-1} + b_n P_{n-2}, \\
Q_n &= a_n Q_{n-1} + b_n Q_{n-2},
\end{align*}
with initial starting conditions $P_{-1} = 1$, $P_0 = a_0$, $Q_{-1} = 0$, and $Q_0 = 1$.

The sequence of convergents $\frac{P_n}{Q_n}$ of the generalized continued fraction for $r$ will approach $r$ as $n\to\infty$. That is,
\begin{align*}
\lim_{n\to\infty} \frac{P_n}{Q_n} = r .
\end{align*}

By the theorem's assumption, we have
\begin{align*}
\frac{P_n}{P_{n-1}} \leq 2^{2^{2^{\ldots^n}}} \text{ and } \frac{Q_n}{Q_{n-1}} \leq 2^{2^{2^{\ldots^n}}} .
\end{align*}
This implies that for all $n \geq 1$,
\begin{align*}
P_n &\leq P_{n-1} 2^{2^{2^{\dots^n}}}
    \leq P_{n-2} 2^{2^{2^{\dots^{n-1}}}} 2^{2^{2^{\dots^n}}}
    \leq \ldots \leq \\
    & P_0 2^{2^{2^1}} 2^{2^{2^2}} \cdots 2^{2^{2^{\dots^n}}}
    = P_0 2^{2^1 + 2^2 + \cdots + 2^{2^{\dots^n}}}
    = P_02^{2^{2^{\dots^{n+1}}} - 2} \\
    &= O(2^{2^{2^{\dots^n}}}),
\end{align*}
and similarly for $Q_n$.

The set of rational numbers $\mathbb{Q}$ is dense in $\mathbb{R}$. Thus, for any $r \in \mathbb{R}$ and any $\epsilon > 0$, there exists a rational number $q \in \mathbb{Q}$ such that $|r - q| < \epsilon$.

Define $\epsilon_n = \left| r - \frac{P_n}{Q_n} \right|$. Since $\lim_{n\to\infty} \frac{P_n}{Q_n} = r$, there exists an $N \in \mathbb{N}$ such that for all $n > N$, we have $\epsilon_n < \frac{P_n}{Q_n}$ and $\epsilon_{n+1} < \epsilon_n$. Furthermore, since $P_n$ and $Q_n$ are bounded by $O(2^{2^{\ldots^{2^n}}})$ and their recursions require only elementary arithmetic operations to compute, there exists a pair of Kalmar functions $f(n)$ and $g(n)$ such that $f(n)=P_n$ and $g(n) = Q_n$. Therefore, we conclude
\begin{align*}
\lim_{n\to\infty} \left| r - \frac{f(n)}{g(n)} \right| = 0 
\implies \lim_{n\to\infty} \frac{f(n)}{g(n)} = r .    
\end{align*}
\end{proof}

\section{Pochhammer Function}
We define the integer-valued function
\begin{align*}
(n)_{a,b} = \prod_{k=0}^n (a+bk)
\end{align*}

\begin{theorem} \label{proof:pochhammer1}
Let $n,a,b \in \Z$ such that $n > 0$.
Set $\ell = \floor{\log_b((n)_{a,b})}+1$, $m = b^{\ell}+1$, and $c = (ab^{-1}) \bmod m$. Then
\begin{align*}
(n)_{a,b} = \left( \binom{n+c}{n+1} b^{n+1} (n+1)! \right) \bmod m .
\end{align*}
Further, $(n)_{a,b}$ is an arithmetic term.
\end{theorem}
\begin{proof}
From \cite{matiyasevich1993hilbert}, for $a \equiv bc \pmod{m}$, we have
\begin{align} \label{formula:matiyasevich1}
\prod_{k=0}^n (a+bk) \equiv \binom{n+c}{n+1} b^{n+1} (n+1)! \pmod{m} .
\end{align}
We are given $m = b^{\ell}+1$, thus $m$ and $b$ are coprime and $b^{-1} \pmod{m}$ exists. Therefore, $c = ab^{-1} \bmod m$ is valid. Hence, we have established a congruence of $(n)_{a,b}$ modulo $m$.

Now, since
\begin{align*}
(n+1)_{a,b} > m = b^{\ell} = b^{\floor{\log_b((n)_{a,b})}+1} > (n)_{a,b} ,
\end{align*}
it follows that
\begin{align*}
(n)_{a,b} = \left( \binom{n+c}{n+1} b^{n+1} (n+1)! \right) \bmod m .
\end{align*}

Due to Prunescu and Sauras-Altuzarra \cite{prunescu2024factorial}, we have an arithmetic term for the modular inverse $b^{-1} \pmod{m}$. Arithmetic terms for the factorial function, binomial coefficients, and roots are also known \cite{robinson1952arithmetic,shunia2023simple,shunia2024polynomial, prunescu2024factorial}. Therefore, it follows that $(n)_{a,b}$ is an arithmetic term.
\end{proof}

\section{Gamma Function}
\begin{theorem}
The function $\Gamma_k(q)$, where $q \in \mathbb{Q}$, is a Kalmar number.
\end{theorem}
\begin{proof}
From \cite{diaz2005hypergeometric}, we have
\begin{align*}
\Gamma_k(q) = \lim_{n\to\infty} \frac{n! k^n (nk)^{q/k-1}}{(q)_{n,k}} .
\end{align*}
By \cref{proof:pochhammer1}, $(q)_{n,k}$ is an arithmetic term. Arithmetic terms for the factorial function and roots are also known \cite{prunescu2024factorial,shunia2024polynomial}. Therefore, it follows that $\Gamma_k(q)$ is a Kalmar number.
\end{proof}

\section{Polygamma Function}
From our Kalmar number for $\Gamma(q)$, we can calculate $\psi(q)$.
\begin{align*}
\psi^{(0)}(q) =\psi(q) = \frac{\Gamma'(q)}{\Gamma(q)} .
\end{align*}

\section{Useful Kalmar Numbers}

\textbf{Roots}
\begin{align}
\sqrt[n]{a} &= \lim_{k\to\infty}
    \frac{(k^{kn} + 1)^{kn+1} \bmod (k^{kn^2}-a)}
    {(k^{kn} + 1)^{kn} \bmod (k^{kn^2}-a)} - 1 .
\end{align}

\textbf{Derivatives}
\begin{align}
f'(x) = \lim_{b \to 0} \frac{f(a+b) - f(a)}{b}
\end{align}

\textbf{Exponential Function}
\begin{align*}
    e = \lim_{k\rightarrow\infty} \frac{(k+1)^k}{k^k}
\end{align*}
\begin{align*}
    e^n = \lim_{k\rightarrow\infty} \frac{(k+1)^{nk}}{k^{nk}}
\end{align*}
\begin{align*}
\exp(in) = \lim_{k\to\infty} \left(\left(1+\frac{xn}{k} \right)^k \bmod (x^2+1) \right)
\end{align*}

\textbf{Natural Logarithm}
\begin{align*}
\log(n) = \lim_{k\rightarrow\infty} k (\sqrt[k]{n}-1) .
\end{align*}

\textbf{Trigonometric Functions}
\begin{align*}
\cos(n) &= \lim_{k\to\infty} \left( \left( \left( 1+\frac{xn}{k} \right)^k \bmod (x^2+1) \right) \bmod x \right) , \\
\sin(n) &= \lim_{k\to\infty} \floor{\frac{\left( 1+\frac{xn}{k} \right)^k \bmod (x^2+1)}{x}} .
\end{align*}

\textbf{Mod One}
\begin{align*}
a \bmod b
= b \left( \frac{a}{b} \bmod 1 \right)
= b \left( \frac{a}{b} - \floor{\frac{a}{b}} \right)
\end{align*}

\textbf{Pi}
\begin{align*}
\sqrt{2\pi} = \lim_{n\rightarrow\infty} \frac{n!}{\sqrt{n}n^n e^{-n}}
\end{align*}
\begin{align*}
\sqrt{\pi} = \lim_{n\rightarrow\infty} \frac{4^n n!^2}{\sqrt{n} (2n)!}
\end{align*}
\begin{align*}
\frac{\pi}{2}
= \lim_{n\rightarrow\infty} \frac{2^{4n} n!^4}{(2n)!^2 (2n+1)}
\end{align*}

%%%%%%%%%%%%%%%%%%%%%%%%%%%%%%%%%%%%%%%%%%%%%%%%%%%%%%%%%%%%%%%%%%%%
\footnotesize

\begingroup
\raggedright
\bibliographystyle{plainnat}
\bibliography{main}
\endgroup

\end{document}