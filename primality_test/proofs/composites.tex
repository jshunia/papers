\pdfoutput=1
\documentclass{article}
\usepackage{fullpage}
\usepackage{amsmath, amssymb, amsthm}
\usepackage[utf8]{inputenc}
\usepackage[english]{babel}
\usepackage[numbers]{natbib}
\usepackage{csquotes}
\usepackage{url}
\usepackage{hyperref}
\usepackage{cleveref}
\usepackage{algorithm}
\usepackage{algpseudocode}
\usepackage{algorithmicx}
\setlength{\parskip}{0.5em}
\setlength{\parindent}{0pt}
\theoremstyle{plain}
\theoremstyle{definition}
\newtheorem{definition}{Definition}
\newtheorem{identity}{Identity}
\newtheorem{conjecture}{Conjecture}
\newtheorem{theorem}{Theorem}
\newtheorem{lemma}{Lemma}
\newtheorem{proposition}{Proposition}
\newtheorem{example}{Example}
\usepackage{graphicx}
\usepackage{multicol}
\newcommand{\floor}[1]{\left\lfloor #1 \right\rfloor}
\newcommand{\totient}[1]{\phi\left({#1}\right)}
\newcommand{\bigO}[1]{O(#1)}
\newcommand{\softO}[1]{\tilde{O}(#1)}
\newcommand{\Z}{\mathbb{Z}}
\newcommand{\F}{\mathbb{F}}
\newcommand{\Zn}{\Z/n}
\newcommand{\Zx}{\Z[x]}
\newcommand{\Znx}{(\Zn)[x]}
\newcommand{\Rx}{R}
\newcommand{\Mx}{(\Zn)[x]/(x^d-2)}
\newcommand{\ord}[2]{\text{ord}_\text{#1}\left((#2)\right)}
\newcommand{\dtotient}[3]{\sigma_{#1}( #2^{\floor{\frac{1}{#3}}} )}
\newcommand{\ordn}[1]{\text{ord}_n\left({#1}\right)}
\newcommand{\indicator}[2]{\substack{#1 \\ #2}}
\newcommand{\glaisher}[2]{\left[ \begin{array}{c} #1 \\ #2 \end{array} \right]}
\newcommand{\kronecker}[2]{\left(#1\mid#2\right)}

\title{Composites Case Proof Attempt}
\author{Joseph M. Shunia}
\date{January 2024}

\begin{document}

\maketitle

\begin{theorem}[Composites case] \label{theorem:composites}
Let $n = pq$ be an odd composite integer $>3$ with $p$ a prime divisor. Let $d$ be the least positive integer $>2$ such that $n \not\equiv 1 \pmod{d}$. Suppose $x^n = 2^{\left\lfloor\frac{n}{d}\right\rfloor} \not\equiv 1 \pmod{n}$, and consider the polynomial $f(x) = (1 + x)^n - (1 + x^n) \in \mathbb{Z}_p[x]$. If $n$ does not have a prime divisor $\leq d$, then $f(x)$ is nonzero when reduced modulo $x^d - 2$.
\end{theorem}
\begin{proof}
Let $p$ be a prime divisor of $n$. Consider the polynomial ring $\mathbb{Z}_p[x]$. We examine the reduction of $ f(x) $ modulo $x^d - 2$, which gives us a polynomial $f(x) \pmod{x^d - 2} \in \mathbb{Z}_p[x]$. After reduction modulo $x^d - 2$, the polynomial $f(x)$ has $\deg(f(x)) = d-1$, and can be written as $f(x) = \sum_{i=0}^{d-1} c_i x^i$.

The condition $x^n = 2^{\left\lfloor\frac{n}{d}\right\rfloor} \not\equiv 1 \pmod{n}$ implies that $x^n = 2^{\left\lfloor\frac{n}{d}\right\rfloor} \not\equiv 1 \pmod{p}$ for at least $1$ prime divisor $p$ of $n$. Recall also that we are given $d$ which does not divide $n$, and hence $p \not= d$. Together, these imply that the powers $x^k$ in $f(x)$ do not behave in a cyclical manner when reduced modulo $p$, and hence, the polynomial $f(x)$ cannot simplify to the zero polynomial due to any cyclical patterns in the exponents. Furthermore, the fact that $x^d = 2$ in our quotient ring, and not $1$, ensures that $x$ is not a $d$th root of unity in $\mathbb{Z}_p$ for any prime $p$ dividing $n$. Hence, $f(x)$ does not exhibit any cyclical reduction that would occur if $x$ were a root of unity.

Assume for contradiction that $f(x)$ is the zero polynomial in $\mathbb{Z}_p[x]/(x^d - 2)$. This would imply that all coefficients $c_i$ are zero in $\mathbb{Z}_p$.

Since $p$ is prime, $\mathbb{Z}_p$ is a finite field and $\mathbb{Z}[x]_p$ is a ring over this field. Further, since $2^{\left\lfloor\frac{n}{d}\right\rfloor} \not\equiv 1 \pmod{n}$, there must exist at least $1$ prime divisor $p$ of $n$ such that $2^{\left\lfloor\frac{n}{d}\right\rfloor} \not\equiv 1 \pmod{p}$. This implies that $2$ is not a $d$th power residue modulo $p$. That is, $a^d \not\equiv 2 \pmod{p} \in \mathbb{Z}_p$ for all integers $a$. Hence, it follows that $x^d - 2$ is irreducible over $\mathbb{Z}_p[x]$ and therefore, $\mathbb{Z}_p[x]/(x^d - 2)$ forms a field.

By the Fundamental Theorem of Algebra over finite fields, if $\mathbb{Z}_p[x]/(x^d - 2)$ is a field and $\deg(f(x)) = d-1$, then $f(x)$ can have at most $d-1$ roots in $\mathbb{Z}_p[x]/(x^d - 2)$. The assumption that $f(x)$ is zero would imply it has $p$ roots, which is a contradiction unless $p \leq d-1$. However, this is clearly false, since we are given $n$ which does not have a prime divisor $\leq d$.

Therefore, $f(x)$ must be nonzero in $\mathbb{Z}_p[x]/(x^d - 2)$ for at least one prime $p$ that divides $n$.
\end{proof}

\end{document}