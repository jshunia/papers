\pdfoutput=1
\documentclass{article}
\usepackage{fullpage}
\usepackage{amsmath, amssymb, amsthm}
\usepackage[utf8]{inputenc}
\usepackage[english]{babel}
\usepackage[numbers]{natbib}
\usepackage{csquotes}
\usepackage{url}
\usepackage{hyperref}
\usepackage{cleveref}
\usepackage{algorithm}
\usepackage{algpseudocode}
\usepackage{algorithmicx}
\setlength{\parskip}{0.5em}
\setlength{\parindent}{0pt}
\theoremstyle{plain}
\theoremstyle{definition}
\newtheorem{definition}{Definition}
\newtheorem{identity}{Identity}
\newtheorem{conjecture}{Conjecture}
\newtheorem{theorem}{Theorem}
\newtheorem{lemma}{Lemma}
\newtheorem{proposition}{Proposition}
\newtheorem{example}{Example}
\usepackage{graphicx}
\usepackage{multicol}
\newcommand{\floor}[1]{\left\lfloor #1 \right\rfloor}
\newcommand{\totient}[1]{\phi\left({#1}\right)}
\newcommand{\bigO}[1]{O(#1)}
\newcommand{\softO}[1]{\tilde{O}(#1)}
\newcommand{\Z}{\mathbb{Z}}
\newcommand{\F}{\mathbb{F}}
\newcommand{\Zn}{\Z/n}
\newcommand{\Zx}{\Z[x]}
\newcommand{\Znx}{(\Zn)[x]}
\newcommand{\Rx}{R}
\newcommand{\Mx}{(\Zn)[x]/(x^d-2)}
\newcommand{\ord}[2]{\text{ord}_\text{#1}\left((#2)\right)}
\newcommand{\dtotient}[3]{\sigma_{#1}( #2^{\floor{\frac{1}{#3}}} )}
\newcommand{\ordn}[1]{\text{ord}_n\left({#1}\right)}
\newcommand{\indicator}[2]{\substack{#1 \\ #2}}
\newcommand{\glaisher}[2]{\left[ \begin{array}{c} #1 \\ #2 \end{array} \right]}
\newcommand{\kronecker}[2]{\left(#1\mid#2\right)}

\title{Composites Case Proof Attempt}
\author{Joseph M. Shunia}
\date{January 2024}

\begin{document}

\maketitle

\begin{theorem}[Composites case] \label{theorem:composites}
\textit{Let $n = pq$ be an odd composite integer greater than $3$, with $p$ a prime divisor. Let $d > 2$ be the least prime integer such that $n \not\equiv 1 \pmod{d}$. Suppose that $2$ is not a $d$-th power residue modulo $p$ and that $n$ is not divisible by any prime $\leq d$. Then $(1 + x)^n \not\equiv 1 + x^n \pmod{n, x^d-2}$.}
\end{theorem}
\begin{proof}
\textbf{Preliminaries:}
\begin{itemize}
    \item Definition of $p$: Under the given conditions, there must exist at least one prime divisor $p$ of $n$ such that $2$ is not a $d$-th power residue modulo $p$. For the steps included in this proof, $p$ always refers to this specific prime divisor of $n$.
    \item Definition of $f(x)$: We define the polynomial $f(x) := (1 + x)^n - (1 + x^n)$.
    \item Definition of $g(x)$: We define the polynomial $g(x) := f(x) \pmod{x^d - 2}$.
\end{itemize}

\textbf{Introduction}:
If the polynomial congruence $(1 + x)^n \equiv 1 + x^n \pmod{n, x^d-2}$ holds, it implies $f(x) \equiv 0 \pmod{n, x^d-2}$. For this to be true, $f(x)$ must also be zero modulo all of the individual prime factors of $n$ up to at least the prime power that appears in the prime factorization of $n$. This is because if any of the polynomial's coefficients are indivisible by any prime power $p^k$ dividing $n$ (and thus any $p^j$ in $\{ p^j \text{ } | \text{ } 1 < j \leq k \}$), then those coefficients cannot be divisible by $n$ (since $n$ is the unique product of its prime factorization).

Hence, to prove the theorem, it suffices to show that under the given conditions, there must exist a prime divisor $p$ of $n$ such that $f(x) \not\equiv 0 \pmod{p, x^d-2}$. Or equivalently, that $f(x)$ is not the zero polynomial in the ring $\mathbb{Z}_p[x]$ when reduced modulo $x^d-2$. This forms the basis for our hypothesis.

\textbf{Hypothesis}: Under the given conditions, there must exist at least one prime divisor $p$ of $n$ such that $(1 + x)^n \not\equiv 1 + x^n \pmod{p, x^d-2}$.

\textbf{Implications:}
If our hypothesis is true, it implies $(1 + x)^n \not\equiv 1 + x^n \pmod{n, x^d-2}$. Which is the result we intend to prove.

\textbf{Step 1. Establishing irreducibility and field structure}:

Since $p$ is prime, $\mathbb{Z}_p$ is a finite field and $\mathbb{Z}_p[x]$ is a polynomial ring over this field. We aim to establish that $\mathbb{Z}_p[x]/(x^d - 2)$ forms a field. To do so, we must show that $x^d - 2$ is irreducible in $\mathbb{Z}_p[x]$.

We reference a classical theorem from field theory (Irreducibility Theorem) \cite{karpilovsky1989fields}:

\textit{Suppose $c \in F$ where $F$ is a field, and $0 < d \in \mathbb{Z}$. The polynomial $x^k - c$ is irreducible over $F$ if and only if $c$ is not a $q$th power in $F$ for any prime $q$ dividing $k$, and $c$ is not in $-4F^4$ when $4$ divides $k$.}

In our case, $F = \mathbb{Z}_p$, $c = 2$, and $k = d$, where $d$ is prime.

Regarding the first criterion, given $d$ is prime, $d$ does not have any divisors $q$, and hence it suffices to check only $d$ itself. We have chosen $p$ such that $2$ is not a $d$-th power residue modulo $p$, which informs that there is no element $b \in F$ such that $b^d \equiv 2 \pmod{p}$. The first criterion is satisfied. Since $d$ is prime, it is not divisible by $4$, and thus second criterion is also satisfied.

By the Irreducibility Theorem, $x^d - 2$ is irreducible in $\mathbb{Z}_p[x]$ and hence, the quotient ring $\mathbb{Z}_p[x]/(x^d - 2)$ forms a finite field.

\textbf{Step 2. Analyzing the reduction of $f(x)$ modulo $x^d - 2$}:

We examine the reduction of $g(x) = f(x) \pmod{x^d - 2} \in \mathbb{Z}[x]$. After reduction modulo $x^d - 2$, the polynomial $g(x)$ has $\deg(g(x)) = d-1$, and can be written as:
\begin{align}
    g(x) = \sum_{i=0}^{d-1} c_i x^i
\end{align}

To justify this: We first look to the expansion of  $f(x) = (1 + x)^n - (1 + x^n) \in \mathbb{Z}[x]$:
\begin{align}
    f(x) = \sum_{k=1}^{n-1} \binom{n}{k} x^k
\end{align}

Notice that subtracting $1+x^n$ from $(1+x)^n$ cancels out the terms $\binom{n}{0} x^0 = 1$ and $\binom{n}{n} x^n = x^n$ that would typically be present in the binomial expansion of $(1+x)^n$. Thus, we have $\deg(f(x))=n-1$.

Reducing $f(x)$ modulo $x^d - 2$ means replacing every term of the form $\binom{n}{k} x^k$ for $k \geq d$ with a lower-degree term, using the relation $x^d = 2$. During this reduction, terms in $f(x)$ with degree $1 \leq k < d$ will retain their degrees, as they are unaffected by the modulo operation. Since $d < n$, these terms are always present in the binomial expansion. Further, since the highest possible degree of any reduced terms is also $d-1$, the degree of $g(x)$ remains $d-1$ after the reduction of any additional terms.

To ensure $g(x)$ is nonzero, we must also consider the coefficients of the remainder terms. Since the coefficients of the terms in $f(x)$ are the binomial coefficients in the $n$-th row of Pascal's Triangle from $\binom{n}{1}$ to $\binom{n}{n-1}$, it is not possible for all coefficients to be zero after the reduction modulo $x^d - 2$. Instead, the coefficients of these terms will be ``wrapped'' around $x^d - 2$ and added to the fixed term which corresponds to the value of their degree $k$, which is the term with the variable $x^{k \pmod{d}}$. Therefore, after the reduction of $f(x)$ modulo $x^d - 2$, the resultant polynomial $g(x)$ will have $d$ polynomial terms with nonzero coefficients and is not the zero polynomial.

In summary, $x^d-2$ does not divide $f(x)$ in $\mathbb{Z}[x]$ and thus, $g(x)$ is nonzero and has a degree of $d-1$.

\textbf{Step 3. Confirming nonzero polynomial in quotient ring}:

We look to the quotient ring $\mathbb{Z}_p[x]/(x^d - 2)$, which forms a finite field (See Step 1).

In Step 2, we showed that $g(x) = f(x) \pmod{x^d-2}$ is nonzero in $\mathbb{Z}[x]$ with $\deg(g(x)) = d-1$.

To prove our hypothesis, we must also show that $g(x)$ is nonzero in $\mathbb{Z}_p[x]/(x^d - 2)$, as this is equivalent to the statement in our hypothesis, which says: $(1 + x)^n \not\equiv 1 + x^n \pmod{p, x^d-2}$ for at least one prime $p$ dividing $n$.

Now, assume for contradiction that $g(x)$ is the zero polynomial in $\mathbb{Z}_p[x]/(x^d - 2)$. This would necessarily imply that all coefficients $c_i$ of $g(x) = \sum_{i=0}^{d-1} c_i x^i$ are zero when taken modulo $p$, where the $c_i$ are aerated sums of binomial coefficients.

In a finite field, a polynomial of degree $r$ can have at most $r$ roots \cite{dummit2004abstractalgebra}. This is because a polynomial of degree $r$ in a finite field can be factored into at most $r$ linear factors (each corresponding to a root), within an algebraic closure of that field. However, within the field itself, the number of roots can be fewer than $r$, but never more.

In our case, if $\mathbb{Z}_p[x]/(x^d - 2)$ is a finite field and $\deg(g(x)) = d-1$, then $g(x)$ can have at most $d-1$ roots in this field. The assumption that $g(x)$ is the zero polynomial is a direct contradiction unless $p \leq d-1$, as it would necessarily imply that $g(x)$ has infinitely many roots (or more precisely, that every element of $\mathbb{Z}_p$ is a root) \cite{koppartywang2014roots}. However, this is clearly not the case, as we are given $n$ which does not have a prime divisor $\leq d$.

Therefore, $f(x)$ cannot be identically zero in $\mathbb{Z}_p[x]/(x^d - 2)$.

\textbf{Conclusion:}

We have proven our hypothesis under the given conditions by demonstrating $(1 + x)^n \not\equiv 1 + x^n \pmod{p, x^d-2}$ for at least one prime divisor $p$ of $n$. Hence, we deduce $(1 + x)^n \not\equiv 1 + x^n \pmod{n, x^d-2}$. This completes the proof.
\end{proof}

\begingroup
\raggedright
\bibliographystyle{unsrtnat}
\bibliography{main}
\endgroup

\end{document}