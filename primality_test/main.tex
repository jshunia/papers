\pdfoutput=1
\documentclass{article}
\usepackage{fullpage}
\usepackage{amsmath, amssymb, amsthm}
\usepackage[utf8]{inputenc}
\usepackage[english]{babel}
\usepackage[numbers]{natbib}
\usepackage[usenames]{color}
\usepackage{url}
\usepackage{color}
\usepackage[usenames]{color}
\usepackage[usenames]{color}
\usepackage[colorlinks=true,
linkcolor=webgreen,
filecolor=webbrown,
citecolor=webgreen]{hyperref}
\definecolor{webgreen}{rgb}{0,.5,0}
\definecolor{webbrown}{rgb}{.6,0,0}
\definecolor{red}{rgb}{1,0,0}
\usepackage{cleveref}
\usepackage{tabularx}
\usepackage{placeins}
\theoremstyle{plain}
\newtheorem{conjecture}{Conjecture}
\newtheorem*{conjecture*}{Conjecture}
\newtheorem{theorem}{Theorem}[section]
\newtheorem{lemma}[theorem]{Lemma}
\newtheorem{corollary}[theorem]{Corollary}
\newtheorem{proposition}[theorem]{Proposition}
\newtheorem{remark}[theorem]{Remark}
\crefname{conjecture}{Conjecture}{Conjectures}
\crefname{conjecture*}{Conjecture}{Conjectures}
\crefname{theorem}{Theorem}{Theorems}
\crefname{lemma}{Lemma}{Lemmas}
\crefname{proposition}{Proposition}{Propositions}
\crefname{remark}{Remark}{Remarks}
\theoremstyle{definition}
\newtheorem{definition}{Definition}
\newtheorem{notation}{Notation}
\newtheorem{example}{Example}
\newtheorem{question}{Question}
\crefname{definition}{Definition}{Definitions}
\crefname{notation}{Notation}{Notations}
\crefname{example}{Example}{Examples}
\crefname{question}{Question}{Questions}
\crefname{section}{\S}{Sections}
\newcommand{\ideal}[1]{\left\langle #1 \right\rangle}
\newcommand{\floor}[1]{\left\lfloor #1 \right\rfloor}
\newcommand{\ceil}[1]{\left\lceil #1 \right\rceil}
\newcommand{\Z}{\mathbb{Z}}
\newcommand{\K}{\mathcal{K}}
\newcommand{\wt}[1]{\#(#1)}
\newcommand{\lcm}{\mathrm{lcm}}
\newcommand{\ord}{\mathrm{ord}}
\newcommand{\eval}[2]{\left . #1 \right|_{#2}}
\newcommand{\seqnum}[1]{\href{https://oeis.org/#1}{\rm \underline{#1}}}
\setlength{\parskip}{0.5em}
\setlength{\parindent}{0pt}

\title{Modifying Agrawal's Conjecture for Efficient Primality Testing}
\author{Joseph M. Shunia}
\date{December 2023 \\ \small Revised: June 2024 (Version 3.2.1) \normalsize}

\begin{document}

\maketitle

\begin{abstract}
We propose a new deterministic primality test based on a modified version of Agrawal's conjecture, a significant yet unresolved conjecture in the field of computational number theory. By slightly altering the conditions of Agrawal's conjecture and leveraging specific number-theoretic properties, we arrive at a deterministic primality test that achieves a time complexity of $\tilde{O}(\log^3 n)$. In closing, we conjecture the possibility of a $\tilde{O}(\log^2 n)$ primality test, which would match the soft-O time complexity of naive integer multiplication.
\end{abstract}

\section{Introduction}
Primality testing has witnessed remarkable advancements over the past several decades \cite{miller1976, baillie1980, rabin1980, wagstaff1982pseudoprimes, aks2002}. A significant breakthrough in this field was the AKS primality test, introduced by Agrawal, Kayal, and Saxena (2002) \cite{aks2002}. The AKS test was the first to offer determinism and polynomial-time complexity, a monumental achievement that resolved a longstanding open question in computational number theory \cite{goldreich2008}. However, despite its theoretical importance, the AKS test has practical limitations due to its relatively high polynomial time complexity, rendering it inefficient for most applications. Agrawal, Kayal, and Saxena gave a time complexity of $\tilde{O}(\log^{12} n)$ for the AKS test \cite{aks2002}. This bound was lowered significantly by Lenstra and Pomerance (2019) to $\tilde{O}(\log^6 n)$ \cite{lenstra2019gaussian}. Despite this reduction, AKS remains impractical and is mostly unused.

In the field of cryptography, the unique properties of prime numbers are widely exploited to create cryptographic primitives. It is often the case that many large primes must be generated in rapid succession \cite{lenstra1987elliptic}. To make these cryptographic operations practical, fast probabilistic primality tests such as the Baillie-PSW primality test (BPSW) \cite{baillie1980} or Miller-Rabin (MR) \cite{miller1976, rabin1980} are used instead of AKS when searching for large primes. Probabilistic primality tests are by definition non-deterministic and may erroneously report a composite integer as being prime. Composite integers which pass a probabilistic primality test are relatively rare and are known as pseudoprimes (PSPs) for the respective test \cite{wagstaff1982pseudoprimes}. When generating primes for cryptographic purposes, probabilistic primality tests are often combined or repeated with different parameters in order to achieve an acceptable error-bound that makes it almost certain that no composite integer will pass. However, reducing the error-bound requires additional compute and increases running-time, creating a trade-off.

We propose a deterministic primality test with polynomial time complexity $\tilde{O}(\log^3 n)$. This efficiency gain opens new avenues for practical applications, particularly in cryptography, where fast and reliable primality testing is desirable \cite{lenstra1987elliptic}. Our test is based on a famous conjecture issued by Manindra Agrawal at the conclusion of the PRIMES is in P paper \cite{aks2002}.
\begin{conjecture*}[Agrawal]
Let $n$ and $r$ be two coprime positive integers such that $n^2 \not\equiv 1 \pmod{r}$. Let $(x^r-1,n)$ is the ideal generated by $x^r-1$ and $n$ in the polynomial ring $\Z[x]$. Suppose the following polynomial congruence holds:
\begin{align*}
    (x - 1)^n \equiv x^n - 1 \pmod{(x^r - 1,n)} .
\end{align*}
Then, $n$ is prime.
\end{conjecture*}
We alter the conditions of Agrawal's conjecture to use the generator $x^r-2$ instead of $x^r-1$, and update the restriction on $r$ so that $r$ is the least odd prime such that $r \nmid n (n-1)$. These changes allow us to leverage specific number theoretic properties to establish determinism, which were previously unavailable.

Through the proof of our main theorem (\cref{section:maintheorem}) and its supporting lemmas (\cref{section:supportinglemmas}), we demonstrate that our modified test is equivalent to checking the polynomial congruence $(x+1)^n \equiv x^n+1 \pmod{n} \in \mathbb{Z}[x]$, which is known to hold for only prime integers $n$ \cite{granville2004primes}.

\textbf{Disclaimer.} While we believe our main theorem and overall approach is correct, there are still some gaps in a supporting proof that require further work. For details, see the disclaimer following the proof of \cref{proof:carmichaelnumbersfail}.

\section{Supporting Lemmas} \label{section:supportinglemmas}
We begin by presenting the supporting lemmas for our main theorem (\cref{section:maintheorem}).

\begin{lemma} \label{proof:carmichaelnumbersfail}
Let $n \in \Z_{>1}$ be a Carmichael number. Hence, $n = p_1 p_2 \cdots p_m$ is odd, composite, and squarefree, where the $p_i$ are distinct odd prime factors. Furthermore, $(p_i-1) \mid (n-1)$ for all $p_i \mid n$. Let $r \in \mathbb{P}_{\geq 3}$ be the least odd prime such that $r \nmid n (n-1)$. Let $(x^r-2, n)$ be the ideal generated by $x^r-2$ and $n$ in the polynomial ring $\Z[x]$. Consider the polynomial $f(x) := (x+1)^n - x^n - 1 \in \Z[x]$.

Suppose $n$ does not have any prime factor less than $4r$ and $x^n \not\equiv x \pmod{(x^r-2,n)}$. Then
\begin{align*}
    f(x) \not\equiv 0 \pmod{(x^r-2, n)} .
\end{align*}
\end{lemma}
\begin{proof}
The assumption $r \nmid n (n-1)$ implies that $r \nmid n$ and $r \nmid (n-1)$. We will begin by briefly showing why this is necessary. Suppose $r \mid n$, therefore $r = p$ where $p \mid n$. Then
\begin{align*}
    x^n \equiv 2 \pmod{(x^r-2, p)}
    \quad \implies \quad
    (x+1)^n \equiv 3 \pmod{(x^r-2, p)} .
\end{align*}
Hence, we have trivially
\begin{align*}
    f(x) \equiv (x+1)^n - x^n - 1 \equiv 0 \pmod{(x^r-2, p)} .
\end{align*}
Next, suppose $r \mid (n-1)$. Then $r \mid (p-1)$ for some $p \mid n$. Since $p$ is prime, $\varphi(p) = p-1$, where $\varphi$ is Euler's totient function. Leading to
\begin{align*}
    x^n \equiv x \pmod{(x^r-2, p)}
    \quad \implies \quad
    (x+1)^n \equiv x+1 \pmod{(x^r-2, p)} .
\end{align*}
Again, it is easy to see that $f(x) \equiv 0 \pmod{(x^r-2, p)}$ in such case.

We will now show that $f(x) \equiv 0 \pmod{(x^r-2, n)}$ leads to a contradiction under the given conditions. Assume, for the sake of contradiction, that
\begin{align*}
    f(x) \equiv (x+1)^n - x^n - 1 \equiv 0 \pmod{(x^r-2, n)} .
\end{align*}
Since the congruence holds mod $(x^r-2, n)$, it must also hold mod $(x^r-2, p)$ for each prime factor $p$ of $n$. Otherwise, $n$ could not divide $f(x)$. Thus, for all primes $p \mid n$, we have
\begin{align*}
f(x) \equiv (x+1)^n - x^n - 1 \equiv (x+1)^p - x^p - 1 \equiv 0 \pmod{(x^r-2, p)} \\
\implies (x+1)^n - x^n \equiv (x+1)^p - x^p \equiv 1 \pmod{(x^r-2, p)} .
\end{align*}

From this, we deduce
\begin{align*}
\left( (x+1)^{n/p} - x^{n/p} \right)^p \equiv 1 \pmod{(x^r-2, p)} .
\end{align*}

Leading to
\begin{align*}
\left( (x+1)^{n/p} - x^{n/p} \right)^p \equiv (x+1)^n - x^n \equiv (x+1)^p - x^p \equiv 1 \pmod{(x^r-2, p)} .
\end{align*}

This also implies
\begin{align*}
    \zeta_p \equiv (x+1)^{n/p} - x^{n/p} \pmod{(x^r-2, p)} ,
\end{align*}
where $\zeta_p$ is a $p$-th root of unity modulo $(x^r-2, p)$. By the Chinese Remainder Theorem (CRT), since the congruences hold mod $(x^r-2, p)$ for each prime factor $p$ of $n$, they also hold mod $(x^r-2, n)$. Thus, we have
\begin{align*}
\zeta_n &\equiv (x+1)^{n/n} - x^{n/n} \pmod{(x^r-2, n)} \\
&\equiv (x+1)^1 - x^1 \pmod{(x^r-2, n)} \\
&\equiv 1 \pmod{(x^r-2, n)} .
\end{align*}
This is consistent with $\zeta_p$ being a trivial $p$-th root of unity modulo $(x^r-2, n)$. That is
\begin{align*}
\zeta_p \equiv 1 \pmod{(x^r-2, p)} .
\end{align*}
Hence, we have
\begin{align*}
\left( (x+1)^{n/p} - x^{n/p} \right)^p \equiv (x+1)^{n/p} - x^{n/p} \equiv (x+1)^n - x^n \equiv (x+1)^p - x^p \equiv 1 \pmod{(x^r-2, p)} .
\end{align*}

Then, for each $p$, we must consider the following mutually exclusive cases:
\begin{enumerate}
\item[(i)] $x^n \equiv x^p \pmod{(x^r-2, p)} \quad\implies\quad (x+1)^n \equiv (x+1)^p \pmod{(x^r-2, p)}$.
\item[(ii)] $x^p \equiv x^{n/p} \pmod{(x^r-2, p)} \quad\implies\quad (x+1)^p \equiv (x+1)^{n/p} \pmod{(x^r-2, p)}$,
\end{enumerate}
Each case, taken individually, allows for $f(x) \equiv 0 \pmod{(x^r-2,p)}$. These cases are mutually exclusive, since satisfying both (i) and (ii) leads to
\begin{align*}
    x^{n/p} \equiv x^p \equiv x^n \pmod{(x^r-2, p)} ,
\end{align*}
implying that $p=r$ and $r \mid n$, contradicting the theorem.

Suppose either case (i) or (ii) is true for all primes $p \mid n$. For $n$, the two cases (i) and (ii) collapse to a single case, since $p$ is replaced by $n$ in the exponents when lifting via the CRT:
\begin{align*}
x^n \equiv x \pmod{(x^r-2, n)}
\quad\implies\quad
(x+1)^n \equiv x+1 \pmod{(x^r-2, n)} .
\end{align*}
However, this is a contradiction, since again, $x^n \not\equiv x \pmod{(x^r-2, n)}$ by assumption in the theorem. Therefore $f(x) \not\equiv 0 \pmod{(x^r-2, n)}$.

To illustrate, notice
\begin{align*}
    x^n \equiv 2^{\floor{n/r}} x^{n \bmod r} \pmod{x^r-2} , \\
    x^p \equiv 2^{\floor{p/r}} x^{p \bmod r} \pmod{x^r-2} , \\
    x^{n/p} \equiv 2^{\floor{n/(rp)}} x^{(n/p) \bmod r} \pmod{x^r-2} .
\end{align*}

Hence, for case (i), we have
\begin{align*}
    2^{\floor{n/r} - \floor{p/r}} \equiv 1 \pmod{p} 
    \quad \Longleftrightarrow \quad
    \floor{n/r} - \floor{p/r} \equiv 0 \pmod{p-1} .
\end{align*}

For case (ii), we have
\begin{align*}
    2^{\floor{n/r} - \floor{n/(pr)}} \equiv 1 \pmod{p} 
    \quad \Longleftrightarrow \quad
    \floor{n/r} - \floor{n/(pr)} \equiv 0 \pmod{p-1} .
\end{align*}
Since $n$ is a Carmichael number and $(p-1) \mid (n-1)$ for all $p \mid n$, either case would imply $r \mid (n-1)$, contradicting the assumption in the theorem that $r \nmid (n-1)$.

Now, suppose cases (i) and (ii) are both false. The polynomial generator $x^r-2$ factors as $(x-\alpha)^r \pmod{p}$, where $\alpha$ is the root of $x^r-2$ in an algebraic closure of $\mathbb{F}_p$. Let $w = r^{-1} \bmod (p-1)$. Since $n$ is a Carmichael number and $r \nmid (n-1)$, it follows that for all $p \mid n$, we have $r \nmid (p-1)$ and thus $\alpha \equiv 2^w \pmod{p}$.

In the ring $\mathbb{F}_p[x]/((x^r-2)(x-\alpha)^r)$, there may exist non-trivial $p$-th roots of unity, such that
\begin{align*}
    \zeta_p \equiv (x+1)^{n/p} - x^{n/p} \pmod{((x^r-2)(x-\alpha)^r, p)} \\
    \implies
    \left( (x+1)^{n/p} - x^{n/p} \right)^p \equiv 1 \pmod{((x^r-2)(x-\alpha)^r, p)} \\
    \implies
    (x+1)^n - x^n \equiv 1 \pmod{((x^r-2)(x-\alpha)^r, p)} .
\end{align*}

The ring $\Z[x]/(x^r-2, p)$ is isomorphic to the direct product of the field $\mathbb{F}_p[x]/(x-\alpha)$ taken $r$ times. In particular,
\begin{align*}
 \Z[x]/(x^r-2, p) \cong \mathbb{F}_p[x]/((x-\alpha)^r) \cong \mathbb{F}_p[x]/(x-\alpha) \times \cdots \times \mathbb{F}_p[x]/(x-\alpha) .
\end{align*}
The ring $\mathbb{F}_p[x]/((x-\alpha)^r)$ is Artinian, with a chain of ideals $(x-\alpha)^1 \supset (x-\alpha)^2 \supset \cdots \supset (x-\alpha)^r$. Furthermore, since $(x-\alpha)^r \equiv 0 \pmod{p}$, it is a nilpotent ring. In this context, $f(x)$ is trivially zero in $\mathbb{F}_p[x]/((x-\alpha)^r)$. Then, to be zero in the ring $\Z[x]/(x^r-2, p)$, the remainder $f(x) \pmod{p}$ must be in the ideal generated by $(x^r-2)(x-\alpha)^r$.

Notice that $\deg((x^r-2)(x-\alpha)^r) = 2r$. If $p < 2r$, then it is plausible to have $f(x) \equiv 0 \pmod{(x^r-2)(x-\alpha)^r),p)}$ since $p \mid \binom{p}{k}$ for all $1 \leq k \leq (p-1)$ and
\begin{align*}
(x+1)^p \equiv \sum_{k=0}^p \binom{p}{k} x^k \pmod{(x^r-2)(x-\alpha)^r} ,
\end{align*}
leading to
\begin{align*}
(x+1)^p \equiv x^p + 1 \pmod{((x^r-2)(x-\alpha)^r,p)} .
\end{align*}
However, this contradicts the assumption in the theorem that $n$ does not have a prime factor less than $4r$. Similarly, suppose we are given $p$ such that $2r \leq p < 4r$. Then, $2^{\floor{p/(2r)}} \equiv 2^0 \equiv 1 \pmod{p}$ and $x^p$ can be itself when reduced mod $((x^r-2)(x-\alpha)^r,p)$, allowing for
\begin{align*}
x^n \equiv x^p \pmod{((x^r-2)(x-\alpha)^r,p)} .
\end{align*}
Then we would have $f(x) \equiv 0 \pmod{(x^r-2)(x-\alpha)^r),p)}$. However, this again contradicts the assumption in the theorem that $n$ does not have a prime factor less than $4r$.

The only remaining possibility is that
\begin{align*}
    2^{\floor{n/(2r)} - w} \equiv 2^{\floor{n/r}} \pmod{p}
    \implies 2^{\floor{n/(2r)} + w} \equiv 1 \pmod{p} .
\end{align*}
Recall $w \equiv r^{-1} \pmod{p-1}$. Consider the implication, which is
\begin{align*}
    \floor{n/(2r)} \equiv -(r^{-1}) \pmod{p-1} .
\end{align*}
Multiplying both sides by $r$, we see
\begin{align*}
    r \floor{n/(2r)}  \equiv -r r^{-1} \pmod{p-1} \\
    r \floor{n/(2r)} \equiv -1 \pmod{p-1} .
\end{align*}
Lifting mod $n$ by the CRT, this would imply
\begin{align*}
    r \floor{n/(2r)} \equiv -1 \pmod{n-1} .
\end{align*}
Yet we see this is impossible, since $r \geq 3$ and we have $r \floor{n/(2r)} < (n-2)$. Therefore, under the given conditions
\begin{align*}
 f(x) \not\equiv 0 \pmod{(x^r-2, n)} .
\end{align*}
\end{proof}

\begin{remark}
In our proof of \cref{proof:carmichaelnumbersfail}, if the polynomial generator of the ideal was instead $x^r-1$, as in Agrawal's conjecture, we would have:
\begin{align*}
    x^n \equiv x^{n \bmod r} \pmod{(x^r-1, p)}
\end{align*}
In such case, it is plausible to have $(x+1)^n \equiv x^n + 1 \bmod{(x^r-1, n)}$ under alternative circumstances. However, with the generator $x^r-2$, under the given conditions, the remainder $x^{n \bmod r}$ cannot have a coefficient of $1$ unless $n \bmod r \equiv 1 \pmod{n}$, and thus $x^n \equiv x \pmod{(x^r-2,n)}$. In the case of Agrawal's conjecture, the multiplicative order of $1$ modulo $n$ is trivially $\ord_n 1 = 1$. For that reason, it is not useful eliminating edge cases. This shows why using the generator $x^r-2$ is an important strengthening of the conjecture.
\end{remark}

\textbf{Disclaimer.} While we believe \cref{proof:carmichaelnumbersfail} is true and that our proof captures the essence of why, we acknowledge that there are still some gaps to be filled. Specifically, we have outlined various cases for which $p \mid n$ can satisfy the polynomial congruence mod $p$, and demonstrated how these all lead to a contradiction mod $n$. However, we have not shown concretely that these comprise \textbf{all} of the possible cases the polynomial congruence can hold mod $p$ for $p \mid n$. It remains plausible that alternative cases exist. Using Carmichael number tables computed by Richard Pinch \cite{pinch2007carmichael}, we have computationally verified that these are all the possible cases for all $p \mid n$, where $n$ is a Carmichael number, and $n \leq 10^{21} \approx 2^{70}$. This provides some evidence that our assertions are correct, but it is no substitute for a rigorous theoretical proof. Additionally, the congruence $2^{\floor{n/(2r)} - w} \equiv 2^{\floor{n/r}} \pmod{p}$ was found experimentally and we do not yet have a theoretical justification for why it allows for the polynomial congruence to hold mod $p \mid n$.

\begin{lemma} \label{proof:niscarmichaelnumber}
Let $n, r \in \Z_{>1}$ such that $n$ is odd, $r \geq 3$, and $r \nmid n$. Consider the polynomial
\begin{align*}
f(x) := (x+1)^n - x^n - 1 \in \Z[x] .
\end{align*}
Let $(x^r-2, n)$ be the ideal generated by $x^r-2$ and $n$ in the polynomial ring $\Z[x]$. Suppose
\begin{align*}
f(x) \equiv 0 \pmod{(x^r-2, n)} .
\end{align*}
Then it is necessary, but not sufficient, that $n$ is a Carmichael number.
\end{lemma}
\begin{proof}
Let $p \mid n$ be prime. Consider
\begin{align*}
    f(x) \equiv (x+1)^n - x^n - 1 \equiv 0 \pmod{(x^r-2, p)} .
\end{align*}
Expanding the term $(x+1)^n$ via the Binomial Theorem and simplifying, we see
\begin{align*}
    f(x) = \sum_{k=1}^{n-1} \binom{n}{k} x^k .
\end{align*}
From Lucas Theorem, since $p \not= n$, we know that $p$ cannot divide all $\binom{n}{k}$. Furthermore, since $r \geq 3$, the reduction of $f(x)$ mod $(x^r-2)$ will leave a polynomial remainder of degree $d \geq 1$. In other words, $f(x)$ is not a constant modulo $(x^r-2)$. Then, for $f(x)$ to be zero, it must vanish for all $a \in \Z$ when evaluated modulo $p$. That is, 
\begin{align*}
    \forall a \in \Z, \quad f(a) \equiv (a+1)^n - a^n - 1 \equiv 0 \pmod{p} .
\end{align*}
Re-arranging this, we have
\begin{align*}
    \forall a \in \Z, \quad f(a) \equiv (a+1)^n \equiv a^n + 1 \pmod{p} .
\end{align*}
Thus,
\begin{align*}
    \forall a \in \Z, \quad a^n \equiv a \pmod{p} .
\end{align*}
By Fermat's Little Theorem (FLT), this implies $(p-1) \mid (n-1)$. Since this must be true for all primes $p \mid n$, we conclude that $n$ must be a Carmichael number. However, this condition is not sufficient, as we did not prove the conditions under which $f(x) \equiv 0 \pmod{(x^r-2, n)}$ holds for Carmichael numbers.
\end{proof}

\begin{lemma} \label{proof:carmichalenumberxnnotequalsx}
Let $n \in \Z_{>1}$ be a Carmichael number. Hence, $n$ is odd, composite, and squarefree. Let $r \in \mathbb{P}_{\geq 3}$ be the least odd prime such that $r \nmid n (n-1)$. Let $(x^r-2, n)$ be the ideal generated by $x^r-2$ and $n$ in the polynomial ring $\Z[x]$. Then, $x^n \not\equiv x \pmod{(x^r-2,n)}$.
\end{lemma}
\begin{proof}
The reduction of $x^n \pmod{(x^r-2)}$ leaves a polynomial remainder of degree $d = (n \bmod r)$. Formally, we have
\begin{align*}
    x^n \equiv 2^{\floor{n/r}} x^d \equiv 2^{\floor{n/r}} x^{n \bmod r} \pmod{x^r-2}
\end{align*}
Since $n = p_1 p_2 \cdots p_m$ is a Carmichael number, we know that $(p_i-1) \mid (n-1)$ for all prime factors $p_i \mid n$. In such case
\begin{align*}
    r \nmid (n-1) \quad \implies \quad n \not\equiv 1 \pmod{r} . 
\end{align*}
Thus, $d \not= 1$. Leading to
\begin{align*}
    x^n \not\equiv x \pmod{(x^r-2, n)} .
\end{align*}
\end{proof}

\section{Main Theorem} \label{section:maintheorem}
\begin{theorem} \label{proof:main}
Let $n \in \Z_{>1}$ such that $n$ is odd. Let $r \in \mathbb{P}_{\geq 3}$ be the least odd prime such that $r \nmid n (n-1)$. Let $(x^r-2, n)$ be the ideal generated by $x^r-2$ and $n$ in the polynomial ring $\Z[x]$. Consider the polynomial $f(x) := (x+1)^n - x^n - 1 \in \Z[x]$. Suppose $n$ does not have any prime factor less than $4r$ and $f(x) \equiv 0 \pmod{(x^r-2, n)}$. Then, $n$ is prime.
\end{theorem}
\begin{proof}
Expanding $f(x)$ by the Binomial Theorem and simplifying, we see
\begin{align*}
    f(x) = (x+1)^n - x^n - 1 = \sum_{k=1}^{n-1} \binom{n}{k} x^k .
\end{align*}
If $n$ is prime, then $\binom{n}{k} \equiv 0 \pmod{n}$ for all $k$ in the sum. Clearly then, $f(x) \equiv 0 \pmod{n}$ when $n$ is prime.

On the other hand, suppose $n$ is composite. In this case, Lucas Theorem tells us that $n$ cannot possibly divide all $\binom{n}{k}$ in the sum. Applying \cref{proof:niscarmichaelnumber}, we see that $n$ must be a Carmichael number for $f(x)$ to be zero. Furthermore,  by \cref{proof:carmichalenumberxnnotequalsx}, since $r \geq 3$ and $r \nmid (n-1)$, we have $x^n \not\equiv x \pmod{(x^r-2,n)}$. Under these constraints, we have
\begin{align*}
    f(x) \equiv (x+1)^n - x^n - 1 \not\equiv 0 \pmod{(x^r-2, n)} \quad \text{(By \cref{proof:carmichaelnumbersfail})} .
\end{align*}
Therefore, under the given conditions, if $f(x) \equiv 0 \pmod{(x^r-2, n)}$, then $n$ is prime.
\end{proof}

\section{Time Complexity}
\subsection{Proofs}
\begin{lemma} \label{proof:leastcoprime}
Let $n \in \Z_{>6}$. Then, there exists a prime $p$ coprime to $n$, such that
\begin{align*}
    p < \log n \left( \log\log n + \log\log\log n \right)
\end{align*}
\end{lemma}
\begin{proof}
From the Prime Number Theorem, $n$ must have fewer than $\log n$ distinct prime factors. Therefore, we need to establish an upper bound on the $\log n$-th prime number. Let $p_n$ denote the $n$-th prime number. Then, for $n \geq 6$, we have the inequality \cite{rosser1941primebounds}:
\begin{align*}
    \frac{p_n}{n} < \log n + \log\log n .
\end{align*}
Re-arranging terms, we see
\begin{align*}
    p_n < n \log n + n \log\log n .
\end{align*}
Next, we want to establish an upper bound for $p_{\log n}$, the $\log n$-th prime number. Substituting $\log n$ for $n$ in the inequality, we get
\begin{align*}
p_{\log n} < \log n \left( \log\log n + \log\log\log n \right) .
\end{align*}
This expression gives an upper bound for the smallest prime $p$ that is coprime to $n$.
\end{proof}
\begin{theorem} \label{proof:leastcoprimesofto}
Let $n \in \Z_{>6}$. Then, there exists a prime $r = \tilde{O}(\log n)$ such that $r \nmid (n-1)$.
\end{theorem}
\begin{proof}
By \cref{proof:leastcoprime}, there exists a prime integer $r$ coprime to $n-1$, satisfying
\begin{align*}
    r < \log n \left( \log\log n + \log\log\log n \right) .
\end{align*}
Considering the soft-O notation, which simplifies the time complexity by ignoring sublogarithmic factors, we have
\begin{align*}
    \tilde{O}(\log n \left( \log\log n + \log\log\log n \right)) = \tilde{O}(\log n).
\end{align*}
It follows that this bound can be expressed as $\tilde{O}(\log n)$, indicating the existence of an algorithm capable of finding $r$ within this time complexity.
\end{proof}

\section{Algorithm} \label{section:algorithm}
\textbf{INPUT}: An integer $n > 1$.
\begin{center}
    \begin{enumerate}
        \item If $n \equiv 0 \mod{2}$:
            \begin{enumerate}
                \item If $n$ equals $2$, output PRIME.
                \item Otherwise, output COMPOSITE.
            \end{enumerate}
        \item If $n$ equals $3$, output PRIME.
        \item Find the least prime integer $r \geq 3$ such that $n (n-1) \not\equiv 0 \pmod{r}$.
        \item If $n$ has a prime divisor less than $4r$, output COMPOSITE.
        \item Compute the polynomial remainder of $x^n$ in the ring $(\Z/n\Z)[x]/(x^r-2)$ and store the result as $f$. The intermediate terms and polynomial remainder will be of degree at most $r-1$.
        \item Compute the polynomial expansion of $(x+1)^n$ in the ring $(\Z/n\Z)[x]/(x^r-2)$ and store the result as $g$. The intermediate terms and polynomial remainder will be of degree at most $r-1$.
        \item If $g-1 \not= f$, output COMPOSITE.
        \item Output PRIME;
    \end{enumerate}
\end{center}

\subsection{Time Complexity Analysis} \label{subsection:timecomplexity}
The given algorithm is a primality test that involves several computational steps, including modular arithmetic and polynomial exponentiation in the ring $(\Z/n\Z)[x]/(x^r-2)$. In this subsection, we use $M(n)$ to denote the worst-case time complexity of integer multiplication in terms of $n$.

\subsubsection{Analysis of Individual Operations}
\begin{enumerate}
    \item \textbf{Check for Even $n$}:
    
        This step involves calculating $n \bmod{2}$ and has a time complexity of $T_1(n) = O(1)$.

    \item \textbf{Finding $r$}:
    
        Finding the least $r \in \mathbb{P}_{\geq 3}$ such that $n(n-1) \not\equiv 0 \pmod{r}$ takes at most $O(\log n (\log \log n + \log \log \log n))$ steps (\cref{proof:leastcoprimesofto}), with each step requiring $O(1)$ time for the mod operation. Hence, the overall complexity of this stage is $ T_2(n) = O(\log n (\log \log n + \log \log \log n))$.

    \item \textbf{Checking for a prime divisor up to $4r$}:
    
        From the previous stage ($T_2$), we see that $r = O(\log n (\log \log n + \log \log \log n))$. Therefore, this stage requires $O(4 \log n (\log \log n + \log \log \log n))$ steps, with each step requiring $O(1)$ time for the mod operation. Hence, the overall complexity of this stage is $T_3(n) = O(4 \log n (\log \log n + \log \log \log n))$.

    \item \textbf{Computing $x^n \pmod{(x^r-2,n)}$}:
    
        Computing the polynomial expansion of $x^n$ in the ring $(\Z/n\Z)[x]/(x^r-2)$ results in intermediate terms and a polynomial remainder of degree at most $r-1$. This step requires modular exponentiation with a $\log n$-digit base and a $\log n$-digit exponent. The time complexity of modular exponentiation is $ T_4(n) = O(\log n \cdot M(n))$.

    \item \textbf{Computing $(x+1)^n \pmod{(x^r-2,n)}$}:
    
        Computing the polynomial expansion of $(x+1)^n \pmod{n}$ in the ring $(\Z/n\Z)[x]/(x^r-2)$ involves performing modular exponentiation of a polynomial in $(\Z/n\Z)[x]/(x^r-2)$ with $r = T_2(n)$ terms. Exponentiation using repeated squaring takes $O(\log n)$ steps, and each step requires $O(M(n) \cdot T_2(n))$ time due to the multiplication of polynomials of size $T_2(n)$. Therefore, the overall complexity of this stage is $ T_5(n) = O(\log n \cdot  M(n) \cdot  T_2(n))$.
         
    \item \textbf{Checking the congruence $(x+1)^n \equiv x^n+1 \pmod{(x^r-2,n)}$}:
    
    The final steps involve comparing the equality of coefficients in the polynomials $(x+1)^n$ and $x^n+1$. This requires $O(\log n)$ comparisons, which are themselves $O(1)$ operations. The overall time complexity of this stage is $T_6(n) = O(\log n)$.
\end{enumerate}

\subsubsection{Overall Time Complexity}
The dominant time complexity in the algorithm comes from computing $(x+1)^n \pmod{(x^r-2,n)}$, which is $T_5(n)$. Therefore, the overall time complexity of the algorithm is $T(n) = O(\log n \cdot M(n) \cdot T_2(n))$.

Harvey and van Der Heoven (2021) \cite{harveyvanderhoeven2021} have given an algorithm for integer multiplication which has a time complexity $M(n) = O(\log n \log\log n)$. This would give our algorithm an overall time complexity of
\begin{align*}
    T(n) &= O(\log n \cdot  M(n) \cdot  T_2(n))
    \\ &= O(\log n \log n \log\log n \cdot  T_2(n))
    \\ &= O(\log^2 n \log\log n \cdot  T_2(n)) .
\end{align*}

In soft-O notation \cite{gathengerhard2013softo}, typically denoted as $\tilde{O}$, simplicity is achieved by omitting slower-growing logarithmic and lower-order factors that do not significantly contribute to the overall growth rate of the function.

In the context of our overall time complexity $T(n) = O(\log^2 n \log\log n \cdot T_2(n))$, where the dominant term is $O(\log^2 n)$, the linear factors $O(\log\log n)$ are omitted, and the time complexity simplifies to
\begin{align*}
    \tilde{T}(n) = \tilde{O}(\log^2 n \cdot \tilde{T}_2(n)) .
\end{align*}

We have $T_2(n) = O(\log n (\log \log n + \log \log \log n))$, which in soft-O simplifies to $\tilde{T}_2(n) = \tilde{O}(\log n)$. By substitution, we have our final time complexity
\begin{align*}
    \tilde{T}(n) = \tilde{O}(\log^2 n \log n) = \tilde{O}(\log^3 n) .
\end{align*}

\subsubsection{Conclusion}
The overall complexity $\tilde{O}(\log^3 n)$ is polynomial in the size of $n$ when expressed in terms of bit operations, making the algorithm efficient for large values of $n$.

\subsection{Preliminary Computational Verification}
Using tables calculated by Richard Pinch \cite{pinch2007carmichael}, we have computationally verified that there are no counterexamples to the proposed primality test (\cref{proof:main}) for $n \leq 10^{21} \approx 2^{70}$. Additionally, we have verified that all prime divisors $p$ dividing the Carmichael numbers $n \leq 10^{21}$ obey the conditions precisely as described in \cref{proof:carmichaelnumbersfail}.

Sample open source .NET and Python implementations, along with test data, are available on the author's GitHub page \cite{githubrepo}.

\section{Closing Remarks}
It is tempting to wonder if perhaps $\tilde{O}(\log^3 n)$ is the end of the road for primality testing. However, it the opinion of the author that this is likely not the case. As Manindra Agrawal did in the PRIMES is in P paper \cite{aks2002}, we conclude with a conjecture on primality testing.
\begin{conjecture*}
Let $n \in \mathbb{Z}_{>1}$ such that $n$ is odd. Let $(x^8 - x^2 + 2, n)$ be the ideal generated by $x^8 - x^2 + 2$ and $n$ in the polynomial ring $\Z[x]$. Suppose
\begin{align*}
    (x^4 + x)^n - x^{4n} - x^n \equiv 0 \pmod{(x^8 - x^2 + 2, n)} .
\end{align*}
Then $n$ is prime.
\end{conjecture*}
The time complexity of the conjectured primality test is $\tilde{O}(\log^2 n)$. If true, it would mean that primality testing has the same soft-O time complexity as naive integer multiplication. Indeed, this would be quite surprising. Particularly since it was a longstanding question if the problem could be solved in polynomial time at all \cite{aks2002, granville2004primes}.

To understand why the conjectured test is so effective, consider that the univariate congruence in the conjecture is the same as the bivariate congruence
\begin{align*}
    f(x,y) = (x+y)^n - x^n - y^n \equiv 0 \pmod{(x^2-y^2+2, y^4-x, n)} .
\end{align*}
Now, suppose $n$ is composite. For $(x+y)^n - x^n - y^n$ to be zero modulo the ideal generated by $(x^2-y^2+2, y^4-x, n)$, for all primes $p \mid n$, there must exists a pair of integers $(a,b)$ satisfying
\begin{align*}
    a^2 - b^2 \equiv -2 \pmod{p} \quad \land \quad b^4 \equiv a \pmod{p} . 
\end{align*}
While necessary, the property alone is not sufficient. It is necessary, since if the tuple $(x^2-y^2+2, y^4-x)$ is irreducible mod $p$, the
n the quotient ring $\mathbb{F}_p[x,y]/(x^2-y^2+2, y^4-x)$ forms a finite field. In a finite field, a polynomial cannot have more roots than its degree. Thus, $f(x,y)$ is nonzero mod $p$ and since $p \mid n$, $f(x,y)$ is also nonzero mod $n$. On the other hand, it is easy to see that it always holds for prime $n$, since $f(x,y) = (x+y)^n - x^n - y^n = \sum_{k=1}^{n-1} \binom{n}{k} x^k y^{n-k}$ and prime $n$ will divide all $\binom{n}{k}$ in the sum.

We further conjecture that if composite $n$ is a pseudoprime to the conjectured test, then $n$ must also be a Fermat base-$2$ pseudoprime. Under the assumption this statement is true, we have used the Fermat base-$2$ pseudoprime tables provided by William Galway and computed by Jan Feitsma \cite{pseudoprimestables} to verify that there is no composite $n$ which passes the test for $n \leq 2^{64}$.

While we feel this particular version of the conjecture is likely false, as with Agrawal's conjecture, it seems plausible that a variant can be constructed which is deterministic. Though, we also note that, despite some doubts, Agrawal's conjecture remains unresolved \cite{lenstra2003agrawal, popovych2009agrawal}.

\begingroup
\raggedright
\bibliographystyle{unsrtnat}
\bibliography{main}
\endgroup

\end{document}