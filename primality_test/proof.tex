\title{An Efficient Deterministic Primality Test: Proof}
\author{Joseph M. Shunia}
\date{May 2024}

\begin{theorem}
Let $n \in \mathbb{Z}^+$ be a Carmichael number. Hence, $n = p_1 p_2 \cdots p_m$ is odd, composite, and squarefree, where the $p_i$ are odd prime factors.

Let $r \in \mathbb{Z}^+$ be the least odd prime such that $(n-1) \not\equiv 0 \pmod{r}$.

Consider the polynomial $f(x) := (x+1)^n - x^n - 1 \in \mathbb{Z}[x]$. Suppose $x^n \not\equiv x \pmod{(x^r-2,n)}$. Then
\begin{align*}
f(x) \not\equiv 0 \pmod{(x^r-2, n)}.
\end{align*}
\end{theorem}
\begin{proof}
For the sake of contradiction, suppose $f(x) \equiv 0 \pmod{(x^r-2, n)}$.

Since the congruence holds mod $(x^r-2, n)$, it must also hold mod $(x^r-2, p)$ for each prime factor $p$ of $n$. Thus, for all primes $p \mid n$, we have
\begin{align*}
f(x) &\equiv (x+1)^n - x^n - 1 \equiv (x+1)^p - x^p - 1 \equiv 0 \pmod{(x^r-2, p)} \\
\Longleftrightarrow & (x+1)^n - x^n \equiv (x+1)^p - x^p \equiv 1 \pmod{(x^r-2, p)}
\end{align*}

From this, we deduce
\begin{align*}
\left( (x+1)^{n/p} - x^{n/p} \right)^p \equiv 1 \pmod{(x^r-2, p)}
\end{align*}

Leading to
\begin{align*}
\left( (x+1)^{n/p} - x^{n/p} \right)^p \equiv (x+1)^n - x^n \equiv (x+1)^p - x^p \equiv 1 \pmod{(x^r-2, p)}
\end{align*}

This implies
\begin{align*}
    \zeta_p \equiv (x+1)^{n/p} - x^{n/p} \pmod{(x^r-2, p)} ,
\end{align*}
where $\zeta_p$ is a $p$th root of unity.

By the Chinese Remainder Theorem (CRT), since the congruences hold mod $(x^r-2, p)$ for each prime factor $p$ of $n$, they also hold mod $(x^r-2, n)$. Thus, we have
\begin{align*}
\zeta_n &\equiv (x+1)^{n/n} - x^{n/n} \pmod{(x^r-2, n)} \\
&\equiv (x+1)^1 - x^1 \pmod{(x^r-2, n)} \\
&\equiv (x+1) - x \pmod{(x^r-2, n)} \\
&\equiv 1 \pmod{(x^r-2, n)} .
\end{align*}
Leading us back to the possibility
\begin{align*}
\zeta_p \equiv (x+1)^{n/p} - x^{n/p} \equiv (x+1)^n - x^n \equiv (x+1)^p - x^p \equiv 1 \pmod{(x^r-2, p)} .
\end{align*}

Then, for each $p$, we must consider the following cases:
\begin{enumerate}
\item[(i)] $x^n \equiv x^{n/p} \pmod{(x^r-2, p)} \quad\Longleftrightarrow\quad (x+1)^{n} \equiv (x+1)^{n/p} \pmod{(x^r-2, p)}$,
\item[(ii)] $x^p \equiv x^{n/p} \pmod{(x^r-2, p)} \quad\Longleftrightarrow\quad (x+1)^p \equiv (x+1)^{n/p} \pmod{(x^r-2, p)}$,
\item[(iii)] $x^n \equiv x^p \pmod{(x^r-2, p)} \quad\Longleftrightarrow\quad (x+1)^n \equiv (x+1)^p \pmod{(x^r-2, p)}$.
\end{enumerate}
Each case, taken individually, allows for $f(x) \equiv 0 \pmod{(x^r-2,p)}$. A prime $p$ may satisfy one or all cases, since any two cases being true implies the third.

For $n$, the three cases (i), (ii), (iii) collapse to a single case, since $p$ is replaced by $n$ in the exponents when lifting via the CRT:
\begin{align*}
x^n \equiv x \pmod{(x^r-2, n)} \quad\Longleftrightarrow\quad (x+1)^n \equiv x+1 \pmod{(x^r-2, n)}
\end{align*}
However, this is a contradiction, since $x^n \not\equiv x \pmod{(x^r-2, n)}$ by assumption in the theorem. Therefore $f(x) \not\equiv 0 \pmod{(x^r-2, n)}$. This completes the proof.
\end{proof}